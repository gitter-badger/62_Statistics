\documentclass[12pt]{article}
\usepackage{pmmeta}
\pmcanonicalname{OrderStatistics}
\pmcreated{2013-03-22 14:33:30}
\pmmodified{2013-03-22 14:33:30}
\pmowner{CWoo}{3771}
\pmmodifier{CWoo}{3771}
\pmtitle{order statistics}
\pmrecord{8}{36111}
\pmprivacy{1}
\pmauthor{CWoo}{3771}
\pmtype{Definition}
\pmcomment{trigger rebuild}
\pmclassification{msc}{62G30}

% this is the default PlanetMath preamble.  as your knowledge
% of TeX increases, you will probably want to edit this, but
% it should be fine as is for beginners.

% almost certainly you want these
\usepackage{amssymb,amscd}
\usepackage{amsmath}
\usepackage{amsfonts}

% used for TeXing text within eps files
%\usepackage{psfrag}
% need this for including graphics (\includegraphics)
%\usepackage{graphicx}
% for neatly defining theorems and propositions
%\usepackage{amsthm}
% making logically defined graphics
%%%\usepackage{xypic}

% there are many more packages, add them here as you need them

% define commands here
\begin{document}
Let $X_1,\ldots,X_n$ be random variables with realizations in $\mathbb{R}$.  Given an outcome $\omega$, order $x_i=X_i(\omega)$ in non-decreasing order so that 
$$x_{(1)}\leq x_{(2)}\leq\cdots\leq x_{(n)}.$$
Note that $x_{(1)}=\operatorname{min}(x_1,\ldots,x_n)$ and $x_{(n)}=\operatorname{max}(x_1,\ldots,x_n)$.  Then each $X_{(i)}$, such that $X_{(i)}(\omega)=x_{(i)}$, is a random variable.  Statistics defined by $X_{(1)},\ldots,X_{(n)}$ are called \emph{order statistics} of $X_1,\ldots,X_n$.  If all the orderings are strict, then $X_{(1)},\ldots,X_{(n)}$ are \emph{the} order statistics of $X_1,\ldots,X_n$.  Furthermore, each $X_{(i)}$ is called the $i$th order statistic of $X_1,\ldots,X_n$.
\par
\textbf{Remark.}
If $X_1,\ldots,X_n$ are iid as $X$ with probability density function $f_X$ (assuming $X$ is a continuous random variable), Let $\textbf{Z}$ be the vector of the order statistics $(X_{(1)},\ldots,X_{(n)})$ (with strict orderings), then one can show that the joint probability density function $f_{\textbf{Z}}$ of the order statistics is:
$$f_{\textbf{Z}}(\boldsymbol{z})=n!\prod_{i=1}^{n}f_X(z_i),$$
where $\boldsymbol{z}=(z_1,\ldots,z_n)$ and $z_1<\cdots<z_n$.
%%%%%
%%%%%
\end{document}
