\documentclass[12pt]{article}
\usepackage{pmmeta}
\pmcanonicalname{ClusterSampling}
\pmcreated{2013-03-22 15:26:17}
\pmmodified{2013-03-22 15:26:17}
\pmowner{CWoo}{3771}
\pmmodifier{CWoo}{3771}
\pmtitle{cluster sampling}
\pmrecord{4}{37285}
\pmprivacy{1}
\pmauthor{CWoo}{3771}
\pmtype{Definition}
\pmcomment{trigger rebuild}
\pmclassification{msc}{62D05}
\pmsynonym{one-stage cluster sampling}{ClusterSampling}
\pmsynonym{two-stage cluster sampling}{ClusterSampling}
\pmdefines{single-stage cluster sampling}
\pmdefines{second-stage cluster sampling}
\pmdefines{subsampling}

\usepackage{amssymb,amscd}
\usepackage{amsmath}
\usepackage{amsfonts}

% used for TeXing text within eps files
%\usepackage{psfrag}
% need this for including graphics (\includegraphics)
%\usepackage{graphicx}
% for neatly defining theorems and propositions
%\usepackage{amsthm}
% making logically defined graphics
%%%\usepackage{xypic}

% define commands here
\begin{document}
Cluster sampling is a method for sampling from a population $P$, in which the smallest units in $P$ are first grouped into bigger units, called \emph{clusters}, or \emph{primary sampling units}, then a sampling procedure is performed based on these clusters.  The units within each cluster are called the \emph{secondary sampling units}.  For example, when an advertisements are sent out to a sample of
potential customers from a population, it is advisable to first group these potential customers into households, before any sample is drawn, so as to avoid any household receiving more than one ad. There are in general two types of cluster sampling:
\begin{enumerate}
\item \emph{single-stage cluster sampling} or \emph{one-stage cluster sampling}: a sample is taken from the primary sampling units, in which all of the secondary sampling units are considered.
\item \emph{second-stage cluster sampling}, \emph{two-stage cluster sampling}, or emph{subsampling}: a sample is taken from the primary sampling units; then within each primary sampling unit, a sample is taken from their secondary sampling units.
\end{enumerate}
%%%%%
%%%%%
\end{document}
