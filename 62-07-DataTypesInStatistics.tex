\documentclass[12pt]{article}
\usepackage{pmmeta}
\pmcanonicalname{DataTypesInStatistics}
\pmcreated{2013-03-22 14:44:27}
\pmmodified{2013-03-22 14:44:27}
\pmowner{CWoo}{3771}
\pmmodifier{CWoo}{3771}
\pmtitle{data types in statistics}
\pmrecord{15}{36377}
\pmprivacy{1}
\pmauthor{CWoo}{3771}
\pmtype{Topic}
\pmcomment{trigger rebuild}
\pmclassification{msc}{62-07}
\pmdefines{response variable}
\pmdefines{explanatory variable}
\pmdefines{continuous variable}
\pmdefines{discrete variable}
\pmdefines{categorical variable}
\pmdefines{nominal variable}
\pmdefines{ordinal variable}
\pmdefines{predictor}
\pmdefines{control variable}
\pmdefines{observation}
\pmdefines{qualitative variable}
\pmdefines{quantitative variable}
\pmdefines{dichotomous}
\pmdefines{polychotomous}

% this is the default PlanetMath preamble.  as your knowledge
% of TeX increases, you will probably want to edit this, but
% it should be fine as is for beginners.

% almost certainly you want these
\usepackage{amssymb,amscd}
\usepackage{amsmath}
\usepackage{amsfonts}
\usepackage{tabls}

% used for TeXing text within eps files
%\usepackage{psfrag}
% need this for including graphics (\includegraphics)
%\usepackage{graphicx}
% for neatly defining theorems and propositions
%\usepackage{amsthm}
% making logically defined graphics
%%%\usepackage{xypic}

% there are many more packages, add them here as you need them

% define commands here
\begin{document}
\PMlinkescapeword{represents}
\PMlinkescapeword{types}
\PMlinkescapeword{terms}
\PMlinkescapeword{type}
\PMlinkescapeword{code}
\PMlinkescapeword{Acc}
\PMlinkescapeword{level}
\PMlinkescapeword{category}
\PMlinkescapeword{real}
\PMlinkescapeword{mean}
\PMlinkescapeword{binary}
\PMlinkescapeword{multinomial}
\PMlinkescapeword{levels}
\PMlinkescapeword{ordinal}
\PMlinkescapeword{transformation}
\PMlinkescapeword{extension}
\PMlinkescapeword{state}
\PMlinkescapeword{weight}

Data drives statistics.  In traditional statistical analysis, data can usually be visualized by a matrix.  Each column in the matrix represents a data variable (slightly different from the mathematical definition of a variable), and each row respesents an observation or outcome, in which case only one data variable is involved, or a vector of observations or outcomes where several data variables are involved.  

The types of data that are being distinguished have to do with the data variables.  Before going into the details, let's begin with an example as a setting.  A statistical analysis is conducted based on an observational study of autombile insurance data during a particular calendar year $Yr$.  A matrix of data is formed with the following data variables being observed:  
\begin{description}
\item[\textbf{\PMlinkescapetext{Acc}}] whether a policy has been involved in an accident during $Yr$, 
\item[\textbf{NumAcc}] number of accidents have a policy been involved in an accident during $Yr$, 
\item[\textbf{Cost}] the total amount of money a policy cost the insurance company during $Yr$, 
\item[\textbf{Gen}] gender of driver, 
\item[\textbf{Mar}] marital status of driver, 
\item[\textbf{Age}] age of driver, 
\item[\textbf{Hist}] number of accidents a driver had prior to year $Yr$,
\item[\textbf{DrvZIP}] zip code location where the driver lives, 
\item[\textbf{AccZIP}] zip code location where the accident happened, 
\item[\textbf{AccSt}] a numerical code corresponding to the state or province where the accident took place (for example, 0=Alabama, 1=Alaska, etc..., 50=Wyoming),
\item[\textbf{Inj}] the extent of the injury sustained during an accident, 
\item[\textbf{VehType}] the type of vehicle in the policy, and finally, 
\item[\textbf{VehWgt}] the weight of the vehicle in the policy.
\end{description}

Now, we are ready to breakdown the data variables.  First, the data variables can be broken down in terms of their uses:
\begin{enumerate}
\item \emph{response variable} or \emph{predicted variable}.  From the above example, \textbf{\PMlinkescapetext{Acc}}, \textbf{NumAcc}, \textbf{Cost} can all be response variables.  These are variables that we are trying to study, and predict.
\item \emph{explanatory variable} or \emph{predictor variable} or \emph{control variable}.  In the example above, given the response variable is \textbf{\PMlinkescapetext{Acc}}, the explanatory variables can be any of the other variables except \textbf{NumAcc}, \textbf{Cost}, and \textbf{Inj}.  Although possibly highly correlated with \textbf{\PMlinkescapetext{Acc}},  \textbf{NumAcc}, \textbf{Cost}, and \textbf{Inj} do not ``explain" why an accident occurs.  In particular, \textbf{Inj} is only valid when there was an accident.
\end{enumerate}
Usually, the response variable(s) $\boldsymbol{y}$ and the explanatory variable(s) $\boldsymbol{x}$ can be related functionally as 
$$\boldsymbol{y}=f(\boldsymbol{x}).$$
\par
A breakdown of data variables in terms of the natures of the variables is as follows:
\begin{enumerate}
\item \emph{categorical variable} or \emph{discrete variable}.  These are data variables whose ranges are countable, often finite.  Any value of a categorical variable is called a \emph{level}, or a \emph{category}.  For example, \textbf{\PMlinkescapetext{Acc}} is a categorical variable whose values are ``Yes'' (to mean that at least an accident occurred during year $Yr$) and ``No'' (to mean otherwise).  A categorical variable whose number of values is two is often called a \emph{binary variable} or a \emph{dichotomous variable}.  A categorical variable that has more than two values is called a \emph{multinomial variable} or a \emph{polychotomous variable}.  \textbf{DrvZip}, \textbf{Inj} (no injury, light, medium, serious injuries, or death), \textbf{VehType} (family sedan, sports coupe, etc...) and \textbf{NumAcc} are examples of a multinomial variable.
\item \emph{continuous variable}.  Any data variable that is not a categorical variable is a continuous variable.  \textbf{Age} and \textbf{VehWgt} are both examples of continuous variables.  In real situations, these continuous variables usually lie within a certain bounded interval or ball (in higher dimensions).  For example, it is safe to say that the range of the variable \textbf{Age} is $[\ 0, 140\ ]$.
\end{enumerate}
\par
In many statistical modeling situations, it is often convenient, sometimes even desirable to change continuous variables to categorical ones, and vice versa.  Discretization is a way to turn a continuous variable into a categorical one.  For example, the continuous variable \textbf{Age} can be turned into a dichotomous variable by the grouping: ``Young'' = Age $\in [0,25]$ and ``Not Young'' = Age $\in (25,140]$.  Another possible grouping rule may be ``Young'' = Age $\in [0,25]$, ``Mature'' = Age $\in (25,55]$ Age  and ``Old'' = Age $\in (55,140]$.  
\par
Conversely, to turn a categorical variable into a continuous one, either the method of extension or transformation, or both, are used.  For example, \textbf{Hist}, the number of prior accidents is a discrete variable taking on non-negative integer values, can be extended to a continuous variable taking on all non-negative real values to suit a certain modeling function $f$, even though non-integral values do not make sense and are not used in actual predictions.  \textbf{AccZIP} can be transformed into a two-dimensional real-valued vector (longitude, latitude), since each (U.S.) zip code corresponds to an area with a unique centroid whose coordinate is measured in longitude and latitude.
\par
Next, data variables can be grouped as whether they are:
\begin{enumerate}
\item \emph{quantitative}.  All variables such as \textbf{Age}, \textbf{NumAcc}, \textbf{Hist}, and \textbf{VehWgt} are quantitative variables since they take on numerical values.  Variable \textbf{AccSt} is not a quantitative variable even though it is numeric in nature, since its values have no intrinsic numerical meanings.  Another possible non-quantitative variable may be \textbf{DrvZIP}.
\item \emph{qualitative}.  Variables like \textbf{Gen}, \textbf{Mar}, \textbf{Inj}, as well as \textbf{AccSt} and \textbf{DrvZIP} are all qualitative variables.
\end{enumerate}
\par
Finally, data variables can be classified in terms of whether they can be ordered or not:
\begin{enumerate}
\item \emph{nominal} variables have no intrinsic ordering structure.  \textbf{Gen} and \textbf{Mar} are such examples, as are \textbf{AccSt}, \textbf{DrvZIP} and \textbf{VehType}.
\item The meaning of \emph{ordinal} variables is self-explanatory.  Usually, numerical variables are ordinal, except when they are multi-dimensional or vectorial.  \textbf{AccZIP}, when transformed into {longitude,latitude}, is not ordinal.  However, fixing any one of the two coordinates turns the other coordinate into an ordinal variable.  An example of a non-numerical ordinal variable is \textbf{Inj}.  Since the levels of \textbf{Inj} can be ranked by their severity, from ``no injury" to ``death'', it is ordinal.
\end{enumerate}
The data variables in the above example is summarized in the following table:
\begin{center}
\begin{tabular}{|c|c|c|c|c|}
\hline
data variable & use & continuity & numerality & ordinality \\
\hline
\textbf{\PMlinkescapetext{Acc}} & response & categorical & quantitative & nominal \\
\hline
\textbf{NumAcc} & response & categorical & quantitative & ordinal \\
\hline
\textbf{Cost} & response & continuous & quantitative & ordinal \\
\hline
\textbf{Gen} & explanatory & categorical & qualitative & nominal \\
\hline
\textbf{Mar} & explanatory & categorical & qualitative & nominal \\
\hline
\textbf{Age} & explanatory & continuous & quantitative & ordinal \\
\hline
\textbf{Hist} & explanatory & categorical & quantitative & ordinal \\
\hline
\textbf{DrvZIP} & explanatory & categorical & qualitative & nominal \\
\hline
\textbf{AccZIP} & explanatory & categorical & qualitative & nominal \\
\hline
\textbf{AccSt} & explanatory & categorical & qualitative & nominal \\
\hline
\textbf{Inj} & explanatory & categorical & qualitative & ordinal \\
\hline
\textbf{VehType} & explanatory & categorical & qualitative & nominal \\
\hline
\textbf{VehWgt} & explanatory & continuous & quantitative & ordinal \\
\hline
\end{tabular}
\end{center}
%%%%%
%%%%%
\end{document}
