\documentclass[12pt]{article}
\usepackage{pmmeta}
\pmcanonicalname{StatisticalModel}
\pmcreated{2013-03-22 14:33:18}
\pmmodified{2013-03-22 14:33:18}
\pmowner{CWoo}{3771}
\pmmodifier{CWoo}{3771}
\pmtitle{statistical model}
\pmrecord{10}{36107}
\pmprivacy{1}
\pmauthor{CWoo}{3771}
\pmtype{Definition}
\pmcomment{trigger rebuild}
\pmclassification{msc}{62A01}
\pmdefines{identifiable parameterization}
\pmdefines{parameter space}

% this is the default PlanetMath preamble.  as your knowledge
% of TeX increases, you will probably want to edit this, but
% it should be fine as is for beginners.

% almost certainly you want these
\usepackage{amssymb,amscd}
\usepackage{amsmath}
\usepackage{amsfonts}

% used for TeXing text within eps files
%\usepackage{psfrag}
% need this for including graphics (\includegraphics)
%\usepackage{graphicx}
% for neatly defining theorems and propositions
%\usepackage{amsthm}
% making logically defined graphics
%%%\usepackage{xypic}

% there are many more packages, add them here as you need them

% define commands here
\begin{document}
Let $\textbf{X}=(X_1,\ldots,X_n)$ be a random vector with a given realization 
$\textbf{X}(\omega)=(x_1,\ldots,x_n)$, where $\omega$ is the outcome (of an observation or an experiment) in the sample space $\Omega$.  A \emph{statistical model} $\mathcal{P}$ based on $\textbf{X}$ is a set of probability distribution functions of $\textbf{X}$: 
$$\mathcal{P}=\lbrace F_{\textbf{X}} \rbrace.$$
If it is known in advance that this family of distributions comes from a set of continuous distributions, the statistical model $\mathcal{P}$ can be equivalently defined as a set of probability density functions:
$$\mathcal{P}=\lbrace f_{\textbf{X}} \rbrace.$$
\par
As an example, a coin is tossed $n$ times and the results are observed.  The probability of landing a head during one toss is $p$.  Assume that each toss is independent of one another.  If $\textbf{X}=(X_1,\ldots,X_n)$ is defined to be the vector of the $n$ ordered outcomes, then a statistical model based on $\textbf{X}$ can be a family of Bernoulli distributions 
$$\mathcal{P}=\lbrace \prod_{i=1}^n p^{x_i}(1-p)^{1-x_i} \rbrace,$$
where $X_i(\omega)=x_i$ and $x_i=1$ if $\omega$ is the outcome that the $i$th toss lands a head and $x_i=0$ if $\omega$ is the outcome that the $i$th toss lands a tail.
\par
Next, suppose $X$ is the number of tosses where a head is observed, then a statistical model based on $X$ can be a family binomial distributions:
$$\mathcal{P}=\lbrace {n\choose x}p^x(1-p)^{n-x} \rbrace,$$
where $X(\omega)=x$, where $\omega$ is the outcome that $x$ heads (out of $n$ tosses) are observed.
\par
A statistical model is usually \emph{parameterized} by a function, called a \emph{parameterization} 
$$\Theta\rightarrow\mathcal{P}\mbox{ given by }\theta\mapsto F_{\theta}\mbox{ so that }\mathcal{P}=\lbrace F_{\theta} \mid \theta\in\Theta \rbrace,$$
where $\Theta$ is called a \emph{parameter space}.  $\Theta$ is usually a subset of $\mathbb{R}^n$.  However, it can also be a function space.
\par
In the first part of the above example, the statistical model is parameterized by $$p\mapsto\prod_{i=1}^n p^{x_i}(1-p)^{1-x_i}.$$
\par
If the parameterization is a one-to-one function, it is called an \emph{identifiable parameterization} and $\theta$ is called a \emph{parameter}.  The $p$ in the above example is a parameter.
%%%%%
%%%%%
\end{document}
