\documentclass[12pt]{article}
\usepackage{pmmeta}
\pmcanonicalname{LevyProcess}
\pmcreated{2013-03-22 16:37:19}
\pmmodified{2013-03-22 16:37:19}
\pmowner{PrimeFan}{13766}
\pmmodifier{PrimeFan}{13766}
\pmtitle{L\'evy process}
\pmrecord{6}{38820}
\pmprivacy{1}
\pmauthor{PrimeFan}{13766}
\pmtype{Definition}
\pmcomment{trigger rebuild}
\pmclassification{msc}{62M09}
\pmsynonym{Levy process}{LevyProcess}

% this is the default PlanetMath preamble.  as your knowledge
% of TeX increases, you will probably want to edit this, but
% it should be fine as is for beginners.

% almost certainly you want these
\usepackage{amssymb}
\usepackage{amsmath}
\usepackage{amsfonts}

% used for TeXing text within eps files
%\usepackage{psfrag}
% need this for including graphics (\includegraphics)
%\usepackage{graphicx}
% for neatly defining theorems and propositions
%\usepackage{amsthm}
% making logically defined graphics
%%%\usepackage{xypic}

% there are many more packages, add them here as you need them

% define commands here

\begin{document}
In probability theory, a L\'evy process, named after the French mathematician Paul Pierre L\'evy is any continuous-time stochastic process that starts at 0, admits c\`adl\`ag (right-continuous with left limits) modification and has ``stationary independent increments''. The most well-known examples are the Wiener process and the Poisson process.

A continuous-time stochastic process assigns a random variable $X_t$ to each point $t \ge 0$ in time. In effect it is a random function of $t$. The increments of such a process are the differences $X_s - X_t$ between its values at different times $t < s$. To call the increments of a process independent means that increments $X_s - X_t$ and $X_u - X_v$ are independent random variables whenever the two time intervals do not overlap and, more generally, any finite number of increments assigned to pairwise non-overlapping time intervals are mutually (not just pairwise) independent. To call the increments stationary means that the probability distribution of any increment $X_s - X_t$ depends only on the length $s  - t$ of the time interval; increments with equally long time intervals are identically distributed.

In the Wiener process, the probability distribution of $X_s - X_t$ is normal with expected value 0 and variance $s - t$.

In the Poisson process, the probability distribution of $X_s - X_t$ is a Poisson distribution with expected value $\lambda(s - t)$, where $\lambda > 0$ is the intensity or rate of the process.

The probability distributions of the increments of any L\'evy process are infinitely divisible. There is a L\'evy process for each infinitely divisible probability distribution.

In any L\'evy process with finite moments, the $n$th moment $\mu_n(t) = E(X_t^n)$ is a polynomial function of $t$; these functions satisfy a binomial identity: $$\mu_n(t + s) = \sum_{k = 0}^n {n \choose k} \mu_k(t) \mu_{n - k}(s).$$

It is possible to characterise all L\'evy processes by looking at their characteristic function.

{\it This entry was adapted from the Wikipedia article \PMlinkexternal{L\'evy process}{http://en.wikipedia.org/wiki/Levy_process} as of January 25, 2007.}
%%%%%
%%%%%
\end{document}
