\documentclass[12pt]{article}
\usepackage{pmmeta}
\pmcanonicalname{CochransTheorem}
\pmcreated{2013-03-22 14:33:01}
\pmmodified{2013-03-22 14:33:01}
\pmowner{CWoo}{3771}
\pmmodifier{CWoo}{3771}
\pmtitle{Cochran's theorem}
\pmrecord{8}{36101}
\pmprivacy{1}
\pmauthor{CWoo}{3771}
\pmtype{Theorem}
\pmcomment{trigger rebuild}
\pmclassification{msc}{62J10}
\pmclassification{msc}{62H10}
\pmclassification{msc}{62E10}
\pmdefines{Fisher's theorem}

\endmetadata

% this is the default PlanetMath preamble.  as your knowledge
% of TeX increases, you will probably want to edit this, but
% it should be fine as is for beginners.

% almost certainly you want these
\usepackage{amssymb,amscd}
\usepackage{amsmath}
\usepackage{amsfonts}

% used for TeXing text within eps files
%\usepackage{psfrag}
% need this for including graphics (\includegraphics)
%\usepackage{graphicx}
% for neatly defining theorems and propositions
%\usepackage{amsthm}
% making logically defined graphics
%%%\usepackage{xypic}

% there are many more packages, add them here as you need them

% define commands here
\begin{document}
Let $\textbf{X}$ be multivariate normally distributed as $\boldsymbol{N_p(0,I)}$ such that 
$$\textbf{X}^{\operatorname{T}}\textbf{X}=\sum_{i=1}^{k}Q_i,$$
where each 
\begin{enumerate}
\item $Q_i$ is a quadratic form
\item $Q_i=\textbf{X}^{\operatorname{T}}\textbf{B}_i\textbf{X}$, where $\textbf{B}_i$ is a $p$ by $p$ square matrix
\item $\textbf{B}_i$ is positive semidefinite
\item $\operatorname{rank}(\textbf{B}_i)=r_i$
\end{enumerate}
Then any two of the following imply the third:
\begin{enumerate}
\item $\sum_{i=1}^{k}r_i=p$
\item each $Q_i$ has a \PMlinkname{chi square distribution}{ChiSquaredRandomVariable} with $r_i$ \PMlinkescapetext{degrees} of freedom, $\chi^2(r_i)$
\item $Q_i$'s are mutually independent
\end{enumerate}
\par
As an example, suppose ${X_1}^2\sim\chi^2(m_1)$ and ${X_2}^2\sim\chi^2(m_2)$.  Furthermore, assume ${X_1}^2\geq {X_2}^2$ and $m_1>m_2$, then $${X_1}^2-{X_2}^2\sim\chi^2(m_1-m_2).$$  This corollary is known as \emph{Fisher's theorem}.
%%%%%
%%%%%
\end{document}
