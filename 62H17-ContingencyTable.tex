\documentclass[12pt]{article}
\usepackage{pmmeta}
\pmcanonicalname{ContingencyTable}
\pmcreated{2013-03-22 14:32:53}
\pmmodified{2013-03-22 14:32:53}
\pmowner{CWoo}{3771}
\pmmodifier{CWoo}{3771}
\pmtitle{contingency table}
\pmrecord{11}{36096}
\pmprivacy{1}
\pmauthor{CWoo}{3771}
\pmtype{Definition}
\pmcomment{trigger rebuild}
\pmclassification{msc}{62H17}

% this is the default PlanetMath preamble.  as your knowledge
% of TeX increases, you will probably want to edit this, but
% it should be fine as is for beginners.

% almost certainly you want these
\usepackage{amssymb,amscd}
\usepackage{amsmath}
\usepackage{amsfonts}
\usepackage{tabls}
\usepackage{multirow}

% used for TeXing text within eps files
%\usepackage{psfrag}
% need this for including graphics (\includegraphics)
%\usepackage{graphicx}
% for neatly defining theorems and propositions
%\usepackage{amsthm}
% making logically defined graphics
%%%\usepackage{xypic}

% there are many more packages, add them here as you need them

% define commands here
\renewcommand\multirowsetup{\centering}
\newlength\LL \settowidth\LL{100}
\begin{document}
\PMlinkescapeword{times}
\PMlinkescapeword{level}
\PMlinkescapeword{type}
\PMlinkescapeword{levels}

Given a random sample of $N$ observations $\textbf{Z}_i=(Y_i,X_{i1},\ldots,X_{ik})$ where 
\begin{enumerate}
\item the response variables $Y_i$ are identically distributed as $Y$
\item $Y$ is categorical in nature (coming from a multinomial distribution)
\item each of the explanatory variables $X_{ij}$ is categorical in nature
\end{enumerate}
Then we can analyze the data by forming a \emph{contingency table}.  The table is customarily formed by labeling the categories for the response across the top, and then the combinations of the levels for each explanatory variable down the left-most columns.  Then the cells are filled with counts or frequencies of occurrences corresponding to the specific explanatory variable level combination to the left and the response to the top.  
\par
The simplest example of a contingency table is where the response variable $Y$ comes from a binomial distribution (with two possible responses $r_1$ and $r_2$) and there is only one explanatory variable $X$, which has only two levels, $A_1$ and $A_2$.  This is an instance of a \emph{2 way contingency table}:
\begin{center}
\begin{tabular}{|c|c|c|}
\hline
 & $r_1$ & $r_2$ \\
\hline
$A_1$ & $n_{11}$ & $n_{12}$ \\
\hline
$A_2$ & $n_{21}$ & $n_{22}$ \\
\hline
\end{tabular}
\end{center}
where $n_{ij}$ corresponds to the count or frequency of level $A_i$ and response $r_j$.
\par
\textbf{Example}
A penny $P$ and a quarter $Q$ are each tossed separately 100 times.  The outcome for each toss is recorded, $H$ for head and $T$ for tail.  The numbers of heads and tails obtained from the tosses are recorded in the following 2 way table:
\begin{center}
\begin{tabular}{|c|c|c|}
\hline
Coin Type & $H$ & $T$ \\
\hline
$P$ & 45 & 55 \\
\hline
$Q$ & 56 & 44 \\
\hline
\end{tabular}
\end{center}
A \emph{3 way contingency table} consists of one response variable and two explanatory variables.  
\par
\textbf{Example}
Four dice are used in an experiment to test whether they are more or less the ``same'' (having the same probability distribution).  Each die comes from a combination of one of two casinos, $C_1$ and $C_2$, made by one of two manufacturers, $M_1$ and $M_2$.  Furthermore, no two dice have the same combination.  Now, each dice is tossed 120 times and the outcomes (1 through 6) are recorded.  We have the following contingency table:
\begin{center}
\begin{tabular}{|c|c|c|c|c|c|c|c|}
\hline
Casino & Manufacturer & 1 & 2 & 3 & 4 & 5 & 6 \\
\hline
\multirow{2}{\LL}{$C_1$}
& $M_1$ & 17 & 21 & 19 & 20 & 22 & 21 \\ 
\cline{2-8}
 & $M_2$ & 22 & 20 & 20 & 19 & 21 & 18 \\
\hline
\multirow{2}{\LL}{$C_2$}
& $M_1$ & 21 & 18 & 18 & 23 & 21 & 19 \\ 
\cline{2-8}
 & $M_2$ & 20 & 19 & 18 & 21 & 21 & 21 \\
\hline
\end{tabular}
\end{center}
The explanatory variables are the casinos ($C$'s) and the manufacturers ($M$'s), and the response variable is the number appearing on the top face of a thrown die ($1$ through $6$).
%%%%%
%%%%%
\end{document}
