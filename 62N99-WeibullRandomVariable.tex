\documentclass[12pt]{article}
\usepackage{pmmeta}
\pmcanonicalname{WeibullRandomVariable}
\pmcreated{2013-03-22 14:26:44}
\pmmodified{2013-03-22 14:26:44}
\pmowner{CWoo}{3771}
\pmmodifier{CWoo}{3771}
\pmtitle{Weibull random variable}
\pmrecord{8}{35960}
\pmprivacy{1}
\pmauthor{CWoo}{3771}
\pmtype{Definition}
\pmcomment{trigger rebuild}
\pmclassification{msc}{62N99}
\pmclassification{msc}{62E15}
\pmclassification{msc}{60E05}
\pmclassification{msc}{62P05}
\pmsynonym{Weibull distribution}{WeibullRandomVariable}
\pmsynonym{Rayleigh distribution}{WeibullRandomVariable}

% this is the default PlanetMath preamble.  as your knowledge
% of TeX increases, you will probably want to edit this, but
% it should be fine as is for beginners.

% almost certainly you want these
\usepackage{amssymb,amscd}
\usepackage{amsmath}
\usepackage{amsfonts}

% used for TeXing text within eps files
%\usepackage{psfrag}
% need this for including graphics (\includegraphics)
%\usepackage{graphicx}
% for neatly defining theorems and propositions
%\usepackage{amsthm}
% making logically defined graphics
%%%\usepackage{xypic}

% there are many more packages, add them here as you need them

% define commands here
\begin{document}
$X$ is a \emph{Weibull random variable} if it has a probability density function, given by
$$f_X(x)=\frac{\gamma}{\alpha}(\frac{x-\mu}{\alpha})^{\gamma-1}
e^{-(\frac{x-\mu}{\alpha})^\gamma}$$
where $\alpha,\gamma,\mu\in\mathbb{R}$, $\alpha,\gamma>0$ and $x\ge\mu$.  $\alpha$ is the \emph{scale parameter}, $\gamma$ is the \emph{shape parameter}, and $\mu$ is the \emph{location parameter}.

Notation for $X$ having a Weibull distribution is $X\sim \mbox{Wei}(\alpha,\gamma,\mu)$.  Usually, the location and scale parameters are dropped by the transformation $$Y=\frac{X-\mu}{\alpha}$$ so that $Y\sim \mbox{Wei}(\gamma):=\mbox{Wei}(1,\gamma,0)$.  The resulting distribution is called the \emph{standard Weibull}, or \emph{Rayleigh distribution}:
$$f_X(x)=\gamma x^{\gamma-1}\operatorname{exp}(-x^\gamma)$$

\textbf{\PMlinkescapetext{Properties}}:
Given a standard Weibull distribution $X\sim \mbox{Wei}(\gamma)$:
\begin{enumerate}
\item 
$\operatorname{E}[X]=\Gamma(\frac{\gamma+1}{\gamma})$, where $\Gamma$ is the gamma function
\item 
Median = $(\operatorname{ln}2)^{\frac{1}{\gamma}}$
\item
Mode $= \begin{cases}
(1-\frac{1}{\gamma})^{1/\gamma} & \mbox{if $\gamma>1$}\\
0 & \mbox{otherwise} \end{cases}$
\item 
$\operatorname{Var}[X]=\Gamma(\frac{\gamma+2}{\gamma})-\Gamma(\frac{\gamma+1}{\gamma})^2$
\item 
$X\sim \mbox{Wei}(\alpha,\gamma,0)$ iff $X^{\gamma}\sim \mbox{Exp}(\alpha^\gamma)$, the exponential distribution with parameter $\alpha^\gamma$
\end{enumerate}

\textbf{Remark.}
The Weibull distribution is often used to model reliability or lifetime of \PMlinkescapetext{products} such as light bulbs.
%%%%%
%%%%%
\end{document}
