\documentclass[12pt]{article}
\usepackage{pmmeta}
\pmcanonicalname{LinearPlexing}
\pmcreated{2013-03-22 15:27:22}
\pmmodified{2013-03-22 15:27:22}
\pmowner{cogent}{10347}
\pmmodifier{cogent}{10347}
\pmtitle{Linear Plexing}
\pmrecord{4}{37306}
\pmprivacy{1}
\pmauthor{cogent}{10347}
\pmtype{Definition}
\pmcomment{trigger rebuild}
\pmclassification{msc}{62J05}

\endmetadata

% this is the default PlanetMath preamble.  as your knowledge
% of TeX increases, you will probably want to edit this, but
% it should be fine as is for beginners.

% almost certainly you want these
\usepackage{amssymb}
\usepackage{amsmath}
\usepackage{amsfonts}

% used for TeXing text within eps files
%\usepackage{psfrag}
% need this for including graphics (\includegraphics)
%\usepackage{graphicx}
% for neatly defining theorems and propositions
%\usepackage{amsthm}
% making logically defined graphics
%%%\usepackage{xypic}

% there are many more packages, add them here as you need them

% define commands here
\begin{document}
We have fixed $\alpha$ as the maximum acceptable probability of commiting a
type $I$ error.  But associated with each normalized test of significance is
$\gamma$, the probability of commiting a type $I$ error.

Notice that as we make $\gamma$ smaller, the smaller the critical region
becomes and thus the larger the value of the test statistic must be for the
null hypothesis to be rejected.

Find $p$, defined as the smallest $\gamma$ for which we can still reject
$H_0$.

Reject $H_0$ if $\alpha < p$.
%%%%%
%%%%%
\end{document}
