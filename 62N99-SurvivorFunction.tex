\documentclass[12pt]{article}
\usepackage{pmmeta}
\pmcanonicalname{SurvivorFunction}
\pmcreated{2013-03-22 14:27:43}
\pmmodified{2013-03-22 14:27:43}
\pmowner{CWoo}{3771}
\pmmodifier{CWoo}{3771}
\pmtitle{survivor function}
\pmrecord{6}{35981}
\pmprivacy{1}
\pmauthor{CWoo}{3771}
\pmtype{Definition}
\pmcomment{trigger rebuild}
\pmclassification{msc}{62N99}
\pmclassification{msc}{62P05}
\pmdefines{survival time}

% this is the default PlanetMath preamble.  as your knowledge
% of TeX increases, you will probably want to edit this, but
% it should be fine as is for beginners.

% almost certainly you want these
\usepackage{amssymb,amscd}
\usepackage{amsmath}
\usepackage{amsfonts}

% used for TeXing text within eps files
%\usepackage{psfrag}
% need this for including graphics (\includegraphics)
%\usepackage{graphicx}
% for neatly defining theorems and propositions
%\usepackage{amsthm}
% making logically defined graphics
%%%\usepackage{xypic}

% there are many more packages, add them here as you need them

% define commands here
\begin{document}
Let $Y$ be a random variable with cumulative probability distribution function $F_Y(y)$.  Then the \emph{survivor function} $S(y)$ is defined to be:
$$S(y) = 1 - F_Y(y) = P(Y\geq y).$$
The random variable $Y$ is often called the \emph{survival time}.

The survivor function is the probability of survival beyond time $Y=y$.

\textbf{Examples.} The three most commonly used distribution functions for survival time are:
\begin{enumerate}
\item \PMlinkname{exponential distribution}{ExponentialRandomVariable}, with $S(y)=\exp(-\gamma y).$
\item Weibull distribution, with $S(y)=\exp(-y^{\gamma})$ using the standard Weibull distribution.
\item extreme-value distribution, with $S(y)=\exp(-\exp(\displaystyle{\frac{y-\alpha}{\beta}})).$
\end{enumerate}
%%%%%
%%%%%
\end{document}
