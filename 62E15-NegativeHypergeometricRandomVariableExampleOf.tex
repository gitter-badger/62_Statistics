\documentclass[12pt]{article}
\usepackage{pmmeta}
\pmcanonicalname{NegativeHypergeometricRandomVariableExampleOf}
\pmcreated{2013-03-22 12:39:04}
\pmmodified{2013-03-22 12:39:04}
\pmowner{aparna}{103}
\pmmodifier{aparna}{103}
\pmtitle{negative hypergeometric random variable, example of}
\pmrecord{4}{32917}
\pmprivacy{1}
\pmauthor{aparna}{103}
\pmtype{Example}
\pmcomment{trigger rebuild}
\pmclassification{msc}{62E15}

\endmetadata

% this is the default PlanetMath preamble.  as your knowledge
% of TeX increases, you will probably want to edit this, but
% it should be fine as is for beginners.

% almost certainly you want these
\usepackage{amssymb}
\usepackage{amsmath}
\usepackage{amsfonts}

% used for TeXing text within eps files
%\usepackage{psfrag}
% need this for including graphics (\includegraphics)
%\usepackage{graphicx}
% for neatly defining theorems and propositions
%\usepackage{amsthm}
% making logically defined graphics
%%%\usepackage{xypic} 

% there are many more packages, add them here as you need them

% define commands here
\begin{document}
Suppose you have 7 black marbles and 10 white marbles in a jar.  You pull marbles until you have 3 black marbles in your hand.  $X$ would represent the number of white marbles in your hand.
\begin{list}{$\star$ }{}
\item The expected value of $X$ would be $E[X] = \frac{Wb}{B+1} = \frac{3(10)}{7+1} = 3.75$
\item The variance of $X$ would be $Var[X] = \frac{Wb(B-b+1)(W+B+1)}{(B+2)(B+1)^2} = \frac{10(3)(7-3+1)(10+7+1)}{(7+2)(7+1)^2} = 1.875$
\item The probability of having 3 white marbles would be $f_X(3) = \frac{{3+b-1 \choose 3}{W+B-b-3 \choose W-3}}{{W+B \choose W}} = \frac{{3+3-1 \choose 3}{10+7-3-3 \choose 10-3}}{{10+7 \choose 10}} = 0.1697$
\end{list}
%%%%%
%%%%%
\end{document}
