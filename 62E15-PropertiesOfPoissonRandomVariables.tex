\documentclass[12pt]{article}
\usepackage{pmmeta}
\pmcanonicalname{PropertiesOfPoissonRandomVariables}
\pmcreated{2013-03-22 18:50:55}
\pmmodified{2013-03-22 18:50:55}
\pmowner{CWoo}{3771}
\pmmodifier{CWoo}{3771}
\pmtitle{properties of Poisson random variables}
\pmrecord{4}{41657}
\pmprivacy{1}
\pmauthor{CWoo}{3771}
\pmtype{Derivation}
\pmcomment{trigger rebuild}
\pmclassification{msc}{62E15}

\endmetadata

\usepackage{amssymb,amscd}
\usepackage{amsmath}
\usepackage{amsfonts}
\usepackage{mathrsfs}

% used for TeXing text within eps files
%\usepackage{psfrag}
% need this for including graphics (\includegraphics)
%\usepackage{graphicx}
% for neatly defining theorems and propositions
\usepackage{amsthm}
% making logically defined graphics
%%\usepackage{xypic}
\usepackage{pst-plot}

% define commands here
\newcommand*{\abs}[1]{\left\lvert #1\right\rvert}
\newtheorem{prop}{Proposition}
\newtheorem{thm}{Theorem}
\newtheorem{ex}{Example}
\newcommand{\real}{\mathbb{R}}
\newcommand{\pdiff}[2]{\frac{\partial #1}{\partial #2}}
\newcommand{\mpdiff}[3]{\frac{\partial^#1 #2}{\partial #3^#1}}
\begin{document}
\begin{prop} If $X_1,X_2$ are independent Poisson random variables with parameters $\lambda_1,\lambda_2$, then $X_1+X_2$ is a Poisson random variable with parameter $\lambda_1+\lambda_2$. \end{prop}
\begin{proof}  Let $X:=X_1+X_2$ and $\lambda:=\lambda_1+\lambda_2$, let us calculate the distribution function of $X$:
\begin{eqnarray*}
F_X(x) &=& P(X\le x) = P(X_1+X_2\le x) = \sum_{i=0}^x P(X_1+X_2=i) \\
&=& \sum_{i=0}^x \sum_{j=0}^i P(X_1=j \mbox{ and } X_2=i-j) = \sum_{i=0}^x \sum_{j=0}^i P(X_1=j)P(X_2=i-j) \\
&=& \sum_{i=0}^x \sum_{j=0}^i \frac{e^{-\lambda_1} \lambda_1^j}{j!} \frac{e^{-\lambda_2} \lambda_2^{i-j}}{(i-j)!} 
= \sum_{i=0}^x \sum_{j=0}^i \frac{e^{-\lambda}}{i!} \binom{i}{j} \lambda_1^j \lambda_2^{i-j} \\
&=& \sum_{i=0}^x \frac{e^{-\lambda}}{i!} \sum_{j=0}^i \binom{i}{j} \lambda_1^j \lambda_2^{i-j} = \sum_{i=0}^x \frac{e^{-\lambda}}{i!} (\lambda_1+\lambda_2)^i = \sum_{i=0}^x \frac{e^{-\lambda}}{i!} \lambda^i.
\end{eqnarray*}
As a result, $X$ is a Poisson random variable with parameter $\lambda$.  Notice that in the fifth equation, we used the assumption that $X_1$ and $X_2$ are independent.
\end{proof}

As a corollary, any sum of independent Poisson random variables is Poisson, with parameter the sum of the parameters from the independent random variables.

\begin{prop} A Poisson random variable is infinitely divisible. \end{prop}
\begin{proof}  Let $X$ be a Poisson random variable with parameter $\lambda$.  Let $n$ be any positive integer.  Let $X_1,\ldots, X_n$ be independent identically distributed Poisson random variables with parameter $\frac{\lambda}{n}$.  Then the sum of these random variables is easily seen to be Poisson, with parameter $\lambda$, and is therefore identically distributed as $X$.
\end{proof}
%%%%%
%%%%%
\end{document}
