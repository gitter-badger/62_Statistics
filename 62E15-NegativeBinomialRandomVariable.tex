\documentclass[12pt]{article}
\usepackage{pmmeta}
\pmcanonicalname{NegativeBinomialRandomVariable}
\pmcreated{2013-03-22 11:54:15}
\pmmodified{2013-03-22 11:54:15}
\pmowner{bgins}{4516}
\pmmodifier{bgins}{4516}
\pmtitle{negative binomial random variable}
\pmrecord{9}{30524}
\pmprivacy{1}
\pmauthor{bgins}{4516}
\pmtype{Definition}
\pmcomment{trigger rebuild}
\pmclassification{msc}{62E15}
\pmclassification{msc}{18E05}
\pmclassification{msc}{18-00}
\pmsynonym{negative binomial distribution}{NegativeBinomialRandomVariable}

\usepackage{amssymb}
\usepackage{amsmath}
\usepackage{amsfonts}
%\usepackage{graphicx}
%%%%%\usepackage{xypic}
\begin{document}
$X$ is a \emph{negative binomial random variable} with parameters $r$ and $p$ if\\
\par
$f_X(x) ={r+x-1 \choose x} p^r (1-p)^x$,     $x=\{0,1,...\}$	\\
\par
Parameters:\\
\par
\begin{list}{$\star$ }{}
\item $r > 0$
\item $p \in [0,1]$
\end{list}
\par
Syntax:\\
\par
$X\sim NegBin(r,p)$\\
\par
Notes:\\
\par
\begin{enumerate}

\item If $r \in \mathbb{N}$, $X$ represents the number of failed Bernoulli trials before the $r$th success. Note that if $r=1$ the variable is a geometric random variable.
\item $E[X] = r \frac{1-p}{p}$
\item $Var[X] = r \frac{1-p}{p^2}$
\item $M_X(t) = (\frac{p}{1 - (1-p)e^t})^r$

\end{enumerate}
%%%%%
%%%%%
%%%%%
%%%%%
\end{document}
