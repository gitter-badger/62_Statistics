\documentclass[12pt]{article}
\usepackage{pmmeta}
\pmcanonicalname{MultivariateGammaFunctioncomplexvalued}
\pmcreated{2013-03-22 14:22:10}
\pmmodified{2013-03-22 14:22:10}
\pmowner{mathpeter}{5480}
\pmmodifier{mathpeter}{5480}
\pmtitle{multivariate gamma function (complex-valued)}
\pmrecord{14}{35854}
\pmprivacy{1}
\pmauthor{mathpeter}{5480}
\pmtype{Definition}
\pmcomment{trigger rebuild}
\pmclassification{msc}{62H10}
%\pmkeywords{Gamma multivariate complex}
\pmdefines{gamma function (multivariate complex)}

% this is the default PlanetMath preamble.  as your knowledge
% of TeX increases, you will probably want to edit this, but
% it should be fine as is for beginners.

% almost certainly you want these
\usepackage{amssymb}
\usepackage{amsmath}
\usepackage{amsfonts}

% used for TeXing text within eps files
%\usepackage{psfrag}
% need this for including graphics (\includegraphics)
%\usepackage{graphicx}
% for neatly defining theorems and propositions
%\usepackage{amsthm}
% making logically defined graphics
%%%\usepackage{xypic}

% there are many more packages, add them here as you need them

% define commands here

\DeclareMathOperator{\Tr}{Tr}
\begin{document}
The complex multivariate gamma function is defined as
\begin{equation}
\tilde{\Gamma}_m(a)=\int_{\mathfrak{A}} e^{-\Tr A}|A|^{a-m} {\rm d}A,
\end{equation}
where $ \mathfrak{A}$ is the set of all $ m \times m$ positive, complex-valued Hermitian matrices, i.e.
\begin{equation}
\mathfrak{A} = \left\{A \in \Bbb{C}^{m \times m} | A = A^H, A > 0\right\}.
\end{equation}
It can also be expressed in terms of the gamma function as follows
\begin{equation}
\tilde{\Gamma}_m(a)=\pi^{{1 \over 2}m(m-1)} \prod\limits_{i=1}^{m}\Gamma(a-i+1).
\end{equation}

\subsection*{Reference}
A. T. James,``Distributions of matrix variates and latent roots derived from normal samples,'' {\it Ann. Math. Statist.}, vol. 35, pp. 475-501, 1964.
%%%%%
%%%%%
\end{document}
