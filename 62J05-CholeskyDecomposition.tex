\documentclass[12pt]{article}
\usepackage{pmmeta}
\pmcanonicalname{CholeskyDecomposition}
\pmcreated{2013-03-22 12:07:38}
\pmmodified{2013-03-22 12:07:38}
\pmowner{gufotta}{12050}
\pmmodifier{gufotta}{12050}
\pmtitle{Cholesky decomposition}
\pmrecord{16}{31287}
\pmprivacy{1}
\pmauthor{gufotta}{12050}
\pmtype{Definition}
\pmcomment{trigger rebuild}
\pmclassification{msc}{62J05}
\pmclassification{msc}{65-00}
\pmclassification{msc}{15-00}
\pmsynonym{Cholesky factorization}{CholeskyDecomposition}
\pmsynonym{matrix square root}{CholeskyDecomposition}
%\pmkeywords{matrix factorizations}
\pmrelated{SquareRootOfPositiveDefiniteMatrix}
\pmdefines{Cholesky triangle}

\endmetadata

\usepackage{amssymb}
\usepackage{amsmath}
\usepackage{amsfonts}
%\usepackage{graphicx}
%%%%\usepackage{xypic}

\begin{document}
\section{Cholesky Decomposition}

A symmetric and positive definite matrix can be efficiently decomposed into a lower and upper triangular matrix. For a matrix of any type, this is achieved by the LU decomposition which factorizes $A = LU$. If $A$ satisfies the above criteria, one can decompose more efficiently into $A=LL^T$ where $L$ is a lower triangular matrix with positive diagonal elements. $L$ is called the \emph{Cholesky triangle}.

To solve $Ax = b$, one solves first $Ly = b$ for $y$, and then $L^Tx=y$ for $x$.

A variant of the Cholesky decomposition is the form $A=R^TR$ , where $R$ is upper triangular.

Cholesky decomposition is often used to solve the normal equations in linear least squares problems; they give $A^TAx=A^Tb$ , in which $A^TA$ is symmetric and positive definite.

To derive $A=LL^T$, we simply equate coefficients on both sides of the equation:

$$ 
\begin{bmatrix}
 a_{11} & a_{12} & \cdots & a_{1n} \\
 a_{21} & a_{22} & \cdots & a_{2n} \\
 a_{31} & a_{32} & \cdots & a_{3n} \\
 \vdots & \vdots & \ddots & \vdots \\
 a_{n1} & a_{n2} & \cdots & a_{nn} 
\end{bmatrix}
=
\begin{bmatrix}
 l_{11} & 0      & \cdots & 0 \\
 l_{21} & l_{22} & \cdots & 0 \\
 \vdots & \vdots & \ddots & \vdots \\
 l_{n1} & l_{n2} & \cdots & l_{nn}
\end{bmatrix}
\begin{bmatrix}
 l_{11} & l_{21} & \cdots & l_{n1} \\
 0      & l_{22} & \cdots & l_{n2} \\
 \vdots & \vdots & \ddots & \vdots \\
 0      & 0      & \cdots & l_{nn}
\end{bmatrix}
$$ 

Solving for the unknowns (the nonzero $l_{ji}$s), for $i=1,\cdots ,n$ and $j=i-1,\ldots ,n$, we get:

\begin{eqnarray*}
 l_{ii} & = & \sqrt{\left( a_{ii} - \sum_{k=1}^{i-1} l_{ik}^2 \right)} \\
 l_{ji} & = & \left(a_{ji} - \sum_{k=1}^{i-1} l_{jk} l_{ik} \right) / l_{ii}
\end{eqnarray*}

Because $A$ is symmetric and positive definite, the expression under the square root is always positive, and all $l_{ij}$ are real. 

\begin{thebibliography}{3}

\bibitem{DAB} Originally from The Data Analysis Briefbook
(\PMlinkexternal{http://rkb.home.cern.ch/rkb/titleA.html}{http://rkb.home.cern.ch/rkb/titleA.html})

\end{thebibliography}

%%%%%
%%%%%
%%%%%
\end{document}
