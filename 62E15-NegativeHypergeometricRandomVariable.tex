\documentclass[12pt]{article}
\usepackage{pmmeta}
\pmcanonicalname{NegativeHypergeometricRandomVariable}
\pmcreated{2013-03-22 12:25:05}
\pmmodified{2013-03-22 12:25:05}
\pmowner{alozano}{2414}
\pmmodifier{alozano}{2414}
\pmtitle{negative hypergeometric random variable}
\pmrecord{16}{32339}
\pmprivacy{1}
\pmauthor{alozano}{2414}
\pmtype{Definition}
\pmcomment{trigger rebuild}
\pmclassification{msc}{62E15}
\pmsynonym{negative hypergeometric distribution}{NegativeHypergeometricRandomVariable}

% this is the default PlanetMath preamble.  as your knowledge
% of TeX increases, you will probably want to edit this, but
% it should be fine as is for beginners.

% almost certainly you want these
\usepackage{amssymb}
\usepackage{amsmath}
\usepackage{amsfonts}

% used for TeXing text within eps files
%\usepackage{psfrag}
% need this for including graphics (\includegraphics)
%\usepackage{graphicx}
% for neatly defining theorems and propositions
%\usepackage{amsthm}
% making logically defined graphics
%%%\usepackage{xypic} 

% there are many more packages, add them here as you need them

% define commands here
\begin{document}
 $X$ is a \textbf{negative hypergeometric random variable} with parameters \textbf{$W, B, b$} if

$f_X(x) = \frac{ { x+b-1 \choose x} {W+B-b-x \choose W-x} }{ {W+B \choose W} }$, $x=\{0,1,...,W\}$

Parameters:

\begin{list}{$\star$ }{}
\item $W \in \{1,2,...\}$
\item $B \in \{1,2,...\}$
\item $b \in \{1,2,...,B\}$
\end{list}

Syntax:

$X\sim NegHypergeo(W,B,b)$

Notes:

\begin{enumerate}
\item $X$ represents the number of ``special'' items (from the $W$ special items) present before the $b$th object from a population with $B$ items.
\item The expected value of $X$ is noted as $E[X] = \frac{Wb}{B+1}$
\item The variance of $X$ is noted as $Var[X] = \frac{Wb(B-b+1)(W+B+1)}{(B+2)(B+1)^2}$
\end{enumerate}

Approximation techniques:

If ${x \choose 2} << W$ and ${b \choose 2} << B$ then $X$ can be approximated as a \textbf{negative binomial random variable} with parameters $r = b$ and $p = \frac{W}{W+B}$.
This approximation simplifies the distribution by looking at a system with replacement for large values of $W$ and $B$.
%%%%%
%%%%%
\end{document}
