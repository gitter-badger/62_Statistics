\documentclass[12pt]{article}
\usepackage{pmmeta}
\pmcanonicalname{HypergeometricRandomVariable}
\pmcreated{2013-03-22 11:54:12}
\pmmodified{2013-03-22 11:54:12}
\pmowner{alozano}{2414}
\pmmodifier{alozano}{2414}
\pmtitle{hypergeometric random variable}
\pmrecord{11}{30523}
\pmprivacy{1}
\pmauthor{alozano}{2414}
\pmtype{Definition}
\pmcomment{trigger rebuild}
\pmclassification{msc}{62E15}
\pmclassification{msc}{81-00}
\pmsynonym{hypergeometric distribution}{HypergeometricRandomVariable}

\usepackage{amssymb}
\usepackage{amsmath}
\usepackage{amsfonts}
\usepackage{graphicx}
%%%%\usepackage{xypic}
\begin{document}
$X$ is a \textbf{hypergeometric random variable} with parameters \textbf{$M, K, n$} if\\
\par
$f_X(x) = \frac{ { K \choose x} {M-K \choose n-x} }{ {M \choose n} }$,     $x=\{0,1,...,n\}$	\\
\par
Parameters:\\
\par
\begin{list}{$\star$ }{}
\item $M \in \{1,2,...\}$
\item $K \in \{0,1,...,M\}$
\item $n \in \{1,2,...,M\}$
\end{list}
\par
Syntax:\\
\par
$X\sim Hypergeo(M,K,n)$\\
\par
Notes:\\
\par
\begin{enumerate}

\item $X$ represents the number of ``special'' items (from the $K$ special items) present on a sample of \PMlinkescapetext{size} $n$ from a population with $M$ items.
\item The expected value of $X$ is noted as $E[X] = n \frac{K}{M}$
\item The variance of $X$ is noted as $Var[X] = n \frac{K}{M} \frac{M-K}{M} \frac{M-n}{M-1}$

\end{enumerate}

Approximation techniques:

If ${K \choose 2} << n, M-K+1-n$ then $X$ can be approximated as a \textbf{binomial random variable} with parameters $n=K$ and $p=\frac{M-K+1-n}{M-K+1}$.
This approximation simplifies the distribution by looking at a system with replacement for large values of $M$ and $K$.
%%%%%
%%%%%
%%%%%
%%%%%
\end{document}
