\documentclass[12pt]{article}
\usepackage{pmmeta}
\pmcanonicalname{Statistic}
\pmcreated{2013-03-22 14:46:18}
\pmmodified{2013-03-22 14:46:18}
\pmowner{CWoo}{3771}
\pmmodifier{CWoo}{3771}
\pmtitle{statistic}
\pmrecord{11}{36416}
\pmprivacy{1}
\pmauthor{CWoo}{3771}
\pmtype{Definition}
\pmcomment{trigger rebuild}
\pmclassification{msc}{62A01}
\pmdefines{sample mean}
\pmdefines{sample variance}

% this is the default PlanetMath preamble.  as your knowledge
% of TeX increases, you will probably want to edit this, but
% it should be fine as is for beginners.

% almost certainly you want these
\usepackage{amssymb,amscd}
\usepackage{amsmath}
\usepackage{amsfonts}

% used for TeXing text within eps files
%\usepackage{psfrag}
% need this for including graphics (\includegraphics)
%\usepackage{graphicx}
% for neatly defining theorems and propositions
%\usepackage{amsthm}
% making logically defined graphics
%%%\usepackage{xypic}

% there are many more packages, add them here as you need them

% define commands here
\begin{document}
A \emph{statistic}, or \emph{sample statistic}, $S$ is simply a function, usually real-valued, of a set of (sample) data or observations $\boldsymbol{X}=(X_1,X_2,\ldots,X_n)$: $S=S(\boldsymbol{X})$.  More formally, let $\Omega$ be the sample space of the data $\boldsymbol{X}$, then $S$ is a function from $\Omega$ to some set $T$, usually a subset of $\mathbb{R}^k$.  The data $\boldsymbol{X}$ is usually considered as a vector of iid random variables $X_i$.
\par
\textbf{Examples}
\begin{enumerate}
\item 100 light bulbs out of 1,000,000 are tested for their functionality.  Then the number $n$, of defective light bulbs in the 100 samples is a statistic.  To see this, define, for each $i$ from 1 to 100, 
$$
x_i = 
\begin{cases}
1 & \text{if the event }X_i=\lbrace \text{the }i\text{th light bulb is defective}\rbrace \\
0 & \text{otherwise.}
\end{cases}
$$
Then $n=\sum_{i=1}^{100} x_i$, a function of the data.  Similarly, the number of operating light bulbs is also a statistic if we switch the 1 and 0 in the above definitions for the $x_i$'s.  If we make all $x_i=1$, then $n$ is just the 
count of the observations, one of the simplest forms of sample statistics.  If we make all $x_i=0$, then $n=0$ is a statistic that is not at all useful.
\item Let $w_1,w_2,\ldots,w_{20}$ be the weights of 20 students from a particular college.  Then the average weight defined by 
$$\overline{w}=\frac{1}{20}\sum_{i=1}^{20}w_i$$
is a statistic.  It is commonly called the \emph{sample mean}.  It is often used to estimate $\operatorname{E}[X]$, the expectation of a particular random variable, which, in this case, is the weight of a student in the college.  Of course, other averages, such as medians, mode, trimmed mean, are also examples of (sample) statistics.
\item Using the same example as in (2), we can define 
$$s^2=\frac{1}{20-1}\sum_{i=1}^{20}(w_i-\overline{w})^2.$$
This is also a statistic, for, after some substitution and rewriting, 
$$s^2=\frac{1}{20-1}\Big[\sum_{i=1}^{20}{w_i}^2-\frac{1}{20}(\sum_{i=1}^{20}{w_i)}^2\Big],$$
which is a function explicitly in terms of the $w_i$'s.  This statistic is known as the \emph{sample variance}, which is a common estimate of $\operatorname{Var}[X]$, the variance of the random variable $X$.  Again, in this example, the $X$ is the weight of a student in the college.
\item Again, borrowing from the same example above, we can simply order the weights of the 20 students in an ascending order, so we get a vector of 20 real numbers $(w_{(1)},w_{(2)},\ldots,w_{(20)})$.  This is also a statistic, called an order statistic.  It is not real-valued and its range is a subset of $\mathbb{R}^{20}$.
\item Given a set of numeric observations $X_1,X_2,\ldots,X_n$, without knowing the distribution of these observations, one can define what is known as the empirical distribution function $\hat{F}$, which is a real-valued function, based on the observations.  This is an example of a statistic whose range is a function space.
\end{enumerate}
\par
\textbf{Remarks}.
\begin{itemize}
\item Any function of a statistic is again a statistic.
\item Since the underlying data is assumed to be random, a statistic is necessarily a random variable.
\item Although mostly real-valued, a statistic can be vector-valued, or even function-valued as we have seen in earlier examples.
\end{itemize}
%%%%%
%%%%%
\end{document}
