\documentclass[12pt]{article}
\usepackage{pmmeta}
\pmcanonicalname{GeometricRandomVariable}
\pmcreated{2013-03-22 11:54:06}
\pmmodified{2013-03-22 11:54:06}
\pmowner{mathcam}{2727}
\pmmodifier{mathcam}{2727}
\pmtitle{geometric random variable}
\pmrecord{14}{30520}
\pmprivacy{1}
\pmauthor{mathcam}{2727}
\pmtype{Definition}
\pmcomment{trigger rebuild}
\pmclassification{msc}{62-00}
\pmclassification{msc}{60-00}
\pmclassification{msc}{92-01}
\pmclassification{msc}{92B05}
\pmsynonym{geometric distribution}{GeometricRandomVariable}

\endmetadata

\usepackage{amssymb}
\usepackage{amsmath}
\usepackage{amsfonts}
\usepackage{graphicx}
%%%%\usepackage{xypic}
\begin{document}
A \textbf{geometric random variable} with parameter $p\in(0,1]$ is one whose density distribution function is given by
\begin{equation*}
f_X(x) = p(1-p)^x,\qquad x=0,1,2,\dotsc
\end{equation*}

This is denoted by $X\sim Geo(p)$.

Notes:
\begin{enumerate}
\item A standard application of geometric random variables is where $X$ represents the number of failed Bernoulli trials before the first success.
\item The expected value of a geometric random variable is given by $E[X] = \frac{1-p}{p}$, and the variance by $Var[X] = \frac{1-p}{p^2}$
\item The moment generating function of a geometric random variable is given by $M_X(t) = \frac{p}{1 - (1-p)e^t}$.

\end{enumerate}
%%%%%
%%%%%
%%%%%
%%%%%
\end{document}
