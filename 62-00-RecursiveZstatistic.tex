\documentclass[12pt]{article}
\usepackage{pmmeta}
\pmcanonicalname{RecursiveZstatistic}
\pmcreated{2013-03-22 19:11:30}
\pmmodified{2013-03-22 19:11:30}
\pmowner{statsCab}{25915}
\pmmodifier{statsCab}{25915}
\pmtitle{Recursive Z-statistic}
\pmrecord{4}{42103}
\pmprivacy{1}
\pmauthor{statsCab}{25915}
\pmtype{Definition}
\pmcomment{trigger rebuild}
\pmclassification{msc}{62-00}

% this is the default PlanetMath preamble.  as your knowledge
% of TeX increases, you will probably want to edit this, but
% it should be fine as is for beginners.

% almost certainly you want these
\usepackage{amssymb}
\usepackage{amsmath}
\usepackage{amsfonts}

% used for TeXing text within eps files
%\usepackage{psfrag}
% need this for including graphics (\includegraphics)
%\usepackage{graphicx}
% for neatly defining theorems and propositions
%\usepackage{amsthm}
% making logically defined graphics
%%%\usepackage{xypic}

% there are many more packages, add them here as you need them

% define commands here

\begin{document}
In respones to: Consider a standard Z-statistic used in hypothesis testing. One of the variables needed to compute the Z-statistic is the number of observations. The problem is that with each additional observation one has to recompute the Z-statistic from scratch. It seems like there is no recursive formulation, e.g. a representation such as
Z(n) = Z(n-1) + new piece of information. Is there perhaps an approximate recursive formulation? Any other thoughts?
Thanks. 

An example hypothesis test is:

$H_0:$ $\mu = \mu_0$

$H_1:$ $\mu \neq \mu_0$

We reject this hypothesis if $\overline{x}$ is either greater than or lower than a critical value.
Assuming the critical values do not change all you have to update is $Z_0$.

The test statistic is: 
$$ Z_0 = \frac{\overline{X} - \mu}{\sigma / \sqrt{n}}$$

Assuming you know $\sigma$, when you get a new variable $X_{n+1}$ you can update $\overline{x}$ 
using $n$, $\overline{X}$, and $X_{n+1}$, then recalculate $Z_0$.

Now if you do not know $\sigma$, and your sample size is large enough to use the Normal distribution, you have
to update your sample variance, $S^2$. If your sample size is not large enough and you are using the t-distribution then
your critical values will change when $n$ changes.

To do update $S$ without recalculating, you should keep running totals of $\sum_i X_i$ and $\sum_i X_i^2$,
so you can update $S$ using the computation formula for the sample variance.




%%%%%
%%%%%
\end{document}
