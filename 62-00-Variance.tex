\documentclass[12pt]{article}
\usepackage{pmmeta}
\pmcanonicalname{Variance}
\pmcreated{2013-03-22 11:53:46}
\pmmodified{2013-03-22 11:53:46}
\pmowner{stevecheng}{10074}
\pmmodifier{stevecheng}{10074}
\pmtitle{variance}
\pmrecord{14}{30510}
\pmprivacy{1}
\pmauthor{stevecheng}{10074}
\pmtype{Definition}
\pmcomment{trigger rebuild}
\pmclassification{msc}{62-00}
\pmclassification{msc}{60-00}
\pmclassification{msc}{81-00}
\pmclassification{msc}{83-00}
\pmclassification{msc}{82-00}
\pmclassification{msc}{55-00}
\pmrelated{GeometricDistribution2}
\pmrelated{StandardDeviation}
\pmrelated{Covariance}
\pmrelated{MeanSquareDeviation}

% The standard font packages
\usepackage{amssymb}
\usepackage{amsmath}
\usepackage{amsfonts}

% For neatly defining theorems and definitions
%\usepackage{amsthm}

% Including EPS/PDF graphics (\includegraphics)
%\usepackage{graphicx}

% Making matrix-based graphics
%%%%%\usepackage{xypic}

% Enumeration lists with different styles
%\usepackage{enumerate}

% Set up the theorem environments
%\newtheorem{thm}{Theorem}
%\newtheorem*{thm*}{Theorem}

\newcommand{\defnterm}[1]{\emph{#1}}

% The standard number systems
\newcommand{\complex}{\mathbb{C}}
\newcommand{\real}{\mathbb{R}}
\newcommand{\rat}{\mathbb{Q}}
\newcommand{\nat}{\mathbb{N}}
\newcommand{\intset}{\mathbb{Z}}

% Absolute values and norms
% Normal, wide, and big versions of the delimeters
\newcommand{\abs}[1]{\lvert#1\rvert}
\newcommand{\absW}[1]{\left\lvert#1\right\rvert}
\newcommand{\absB}[1]{\Bigl\lvert#1\Bigr\rvert}
\newcommand{\norm}[1]{\lVert#1\rVert}
\newcommand{\normW}[1]{\left\lVert#1\right\rVert}
\newcommand{\normB}[1]{\Bigl\lVert#1\Bigr\rVert}

% Inverse functions
\newcommand{\inv}[1]{{#1}^{-1}}

% Differentiation operators
\newcommand{\od}[2]{\frac{d #1}{d #2}}
\newcommand{\pd}[2]{\frac{\partial #1}{\partial #2}}
\newcommand{\pdd}[2]{\frac{\partial^2 #1}{\partial #2}}
\newcommand{\ipd}[2]{\partial #1 / \partial #2}

% Differentials on integrals
\newcommand{\dx}{\, dx}
\newcommand{\dt}{\, dt}
\newcommand{\dmu}{\, d\mu}

% Inner products
\newcommand{\ip}[2]{\langle {#1}, {#2} \rangle}

% Complex numbers
\DeclareMathOperator{\zRe}{Re}
\DeclareMathOperator{\zIm}{Im}
\newcommand{\conjug}[1]{\overline{#1}}

% Calligraphic letters
\newcommand{\sF}{\mathcal{F}}
\newcommand{\sD}{\mathcal{D}}

% Standard spaces
\newcommand{\Hilb}{\mathcal{H}}
\newcommand{\Le}{\mathbf{L}}

% Operators and functions occassionally used in my articles
\DeclareMathOperator{\D}{D}
\DeclareMathOperator{\linspan}{span}
\DeclareMathOperator{\rank}{rank}
\DeclareMathOperator{\lindim}{dim}
\DeclareMathOperator{\sinc}{sinc}

% Probability stuff
\newcommand{\PP}{\mathbb{P}}
\newcommand{\E}{\mathbb{E}}

\DeclareMathOperator{\Var}{Var}
\DeclareMathOperator{\Cov}{Cov}

\begin{document}
\PMlinkescapeword{measure}
\PMlinkescapeword{variation}


\subsection*{Definition}
The \emph{variance} of a real-valued random variable $X$ is
\[
\Var X = \E\bigl[ (X - m)^2 \bigr]\,, \quad m = \E X\,,
\]
provided that both expectations $\E X$ and $\E[(X-m)^2]$ exist.

The variance of $X$ is often denoted by $\sigma^2(X)$, $\sigma^2_X$,
or simply $\sigma^2$.
The exponent on $\sigma$ is put there so that the number 
$\sigma = \sqrt{\sigma^2}$ 
is measured in the same units as the random variable $X$
itself.  

The quantity $\sigma = \sqrt{\Var X}$ is called the \emph{standard deviation}
of $X$; 
because of the compatibility of the measuring units, 
standard deviation is usually the quantity that is quoted
to describe an emprical probability distribution, rather than the variance.

\subsection*{Usage}

The variance is a measure of the dispersion or variation
of a random variable
about its mean $m$.

It is not always the best measure of dispersion for all random variables,
but compared to other measures,
such as the absolute mean deviation, $\E[ \abs{X-m} ]$,
the variance is the most tractable analytically.

The variance is closely related to the $\Le^2$ norm for
random variables over a probability space.

\subsection*{Properties}

\begin{enumerate}
\item
The variance of $X$ is the second moment of $X$ minus 
the square of the first moment:
\[
\Var X  = \E[X^2] - \E[X]^2\,.
\]
This formula is often used to calculate variance analytically.

\item
Variance is not a linear function. It scales quadratically,
and is not affected by shifts in the mean of the distribution:
\[
\Var[ aX + b ] = a^2 \Var X\,, \quad \text{ for any $a, b \in \real$.}
\]

\item
A random variable $X$ is constant almost surely if and only
if $\Var X = 0$.

\item
The variance can also be characterized as
the minimum of expected squared deviation of a random variable from any point:
\[
\Var X = \inf_{a \in \real} \E[(X-a)^2]\,.
\]

\item
For any two random variables $X$ and $Y$ whose variances exist,
the variance of the linear combination $aX + bY$
can be expressed in terms of their covariance:
\[
\Var[aX+bY] = a^2 \Var X  + b^2 \Var Y  + 2ab \Cov[X,Y]\,,
\]
where $\Cov[X,Y] = \E[(X-\E X)(Y-\E Y)]$,
and $a, b \in \real$.

\item
For a random variable $X$, with actual observations $x_1, \dotsc, x_n$,
its variance is often estimated
empirically with the \emph{sample variance}:
\[
\Var X  \approx s^2 = \frac{1}{n-1} \sum_{i=1}^n (x_i - \bar{x})^2\,,
\quad
\bar{x} = \frac{1}{n} \sum_{j=1}^n x_j\,.
\]

\end{enumerate}

%%%%%
%%%%%
%%%%%
%%%%%
\end{document}
