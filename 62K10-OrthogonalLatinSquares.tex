\documentclass[12pt]{article}
\usepackage{pmmeta}
\pmcanonicalname{OrthogonalLatinSquares}
\pmcreated{2013-03-22 16:04:47}
\pmmodified{2013-03-22 16:04:47}
\pmowner{CWoo}{3771}
\pmmodifier{CWoo}{3771}
\pmtitle{orthogonal Latin squares}
\pmrecord{12}{38138}
\pmprivacy{1}
\pmauthor{CWoo}{3771}
\pmtype{Definition}
\pmcomment{trigger rebuild}
\pmclassification{msc}{62K10}
\pmclassification{msc}{05B15}
\pmsynonym{mutually orthogonal Latin squares}{OrthogonalLatinSquares}
\pmsynonym{MOLS}{OrthogonalLatinSquares}
\pmsynonym{pairwise orthogonal Latin squares}{OrthogonalLatinSquares}
\pmdefines{complete set of Latin squares}

\usepackage{amssymb,amscd}
\usepackage{amsmath}
\usepackage{amsfonts}

% used for TeXing text within eps files
%\usepackage{psfrag}
% need this for including graphics (\includegraphics)
%\usepackage{graphicx}
% for neatly defining theorems and propositions
%\usepackage{amsthm}
% making logically defined graphics
%%\usepackage{xypic}
\usepackage{pst-plot}
\usepackage{psfrag}

% define commands here

\begin{document}
\PMlinkescapeword{orthogonal}
\PMlinkescapeword{complete}

Given two Latin squares $L_1=(A,B,C_1,f_1)$ and $L_2=(A,B,C_2,f_2)$ of the same order $n$, we can combine them coordinate-wise to form a single square, whose cells are ordered pairs of elements from $C_1$ and $C_2$ respectively.  Formally, we can form a function $f:A\times B\to C_1\times C_2$ given by $$f(i,j)=(f_1(i,j),f_2(i,j)).$$
This function $f$ says that we have created a new square $A\times B$, whose cell $(i,j)$ contains the ordered pair of values, the first coordinate of which corresponds to the value in cell $(i,j)$ of $L_1$, and the second to the value in cell $(i,j)$ of $L_2$.  We may write the combined square $L_1*L_2$.

For example,

\begin{equation*}
\left(\begin{array}{cccc}
a & b & c & d\\
c & d & a &b\\
d & c & b & a\\
b & a & d & c
\end{array}\right)
*
\left(\begin{array}{cccc}
1 & 2& 3& 4\\
4 & 3 & 2 & 1\\
2 & 1 & 4 & 3\\
3 & 4 & 1 & 2
\end{array}\right)
=
\left(\begin{array}{cccc}
(a,1) & (b,2) & (c,3) & (d,4)\\
(c,4) & (d,3) & (a,2) & (b,1)\\
(d,2) & (c,1) & (b,4) & (a,3)\\
(b,3) & (a,4) & (d,1) & (c,2)
\end{array}\right)
\end{equation*}

In general, the combined square is not a Latin square unless the original two squares are equivalent: $f_1(i,j)=f_1(k,\ell)$ iff $f_2(i,j)=f_2(k,\ell)$.  Nevertheless, the more interesting aspect of pairing up two Latin squares (of the same order) lies in the function $f$:  

\begin{quote}
\textbf{Definition}.  We say that two Latin squares are \emph{orthogonal} if $f$ is a bijection.  
\end{quote}

Since there are $n^2$ cells in the combined square, and $| C_1\times C_2| = n^2$, the function $f$ is a bijection if it is either one-to-one or onto.  It is therefore easy to see that the two Latin squares in the example above are orthogonal.

\textbf{Remarks}.  
\begin{itemize}
\item The combined square is usually known as a \emph{Graeco-Latin square}, originated from statisticians Fischer and Yates.
\item It can be shown that if $L_1,\ldots, L_m$ are Latin squares of order $n\ge 3$ such that each pair of them are orthogonal, then $m\le n-1$.  If the equality occurs, then the set of Latin squares are said to be \emph{complete}.
\item (Bose) If $n\ge 3$, then $L_1,\ldots,L_m$ form a complete set of pairwise orthogonal Latin squares of order $n$ iff there exists a finite projective plane of order $n$.
\end{itemize}

\begin{thebibliography}{9}
\bibitem{ryser} H. J. Ryser, \emph{Combinatorial Mathematics}, The Carus Mathematical Monographs, 1963
\end{thebibliography}
%%%%%
%%%%%
\end{document}
