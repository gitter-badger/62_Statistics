\documentclass[12pt]{article}
\usepackage{pmmeta}
\pmcanonicalname{UnitNormalLossFunction}
\pmcreated{2013-03-22 15:56:42}
\pmmodified{2013-03-22 15:56:42}
\pmowner{georgiosl}{7242}
\pmmodifier{georgiosl}{7242}
\pmtitle{unit normal loss function}
\pmrecord{6}{37955}
\pmprivacy{1}
\pmauthor{georgiosl}{7242}
\pmtype{Definition}
\pmcomment{trigger rebuild}
\pmclassification{msc}{62E15}

% this is the default PlanetMath preamble.  as your knowledge
% of TeX increases, you will probably want to edit this, but
% it should be fine as is for beginners.

% almost certainly you want these
\usepackage{amssymb}
\usepackage{amsmath}
\usepackage{amsfonts}

% used for TeXing text within eps files
%\usepackage{psfrag}
% need this for including graphics (\includegraphics)
%\usepackage{graphicx}
% for neatly defining theorems and propositions
%\usepackage{amsthm}
% making logically defined graphics
%%%\usepackage{xypic}

% there are many more packages, add them here as you need them

% define commands here

\begin{document}
The \emph{\PMlinkescapetext{unit normal loss}} function, $UNL$, is defined by
$$UNL(c)=\int_{c}^{\infty}(t-c)f(t)dt$$
where $c$ is a constant and $f(.)$ is the normal probability distribution function.
\\An alternative computational formula for $UNL$ is the following:
$$UNL(z)=f(z)-z(1-F(z))$$
where $f(.)$ and $F(.)$ are the probability distribution function and cumulative distribution function 
for Standard Normal Distribution   respectively.
\\\textbf{Remark.}
This function has an extensive use in Risk Analysis and the Theory of Blackjack.
%%%%%
%%%%%
\end{document}
