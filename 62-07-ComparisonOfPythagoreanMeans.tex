\documentclass[12pt]{article}
\usepackage{pmmeta}
\pmcanonicalname{ComparisonOfPythagoreanMeans}
\pmcreated{2013-03-22 17:49:01}
\pmmodified{2013-03-22 17:49:01}
\pmowner{pahio}{2872}
\pmmodifier{pahio}{2872}
\pmtitle{comparison of Pythagorean means}
\pmrecord{35}{40280}
\pmprivacy{1}
\pmauthor{pahio}{2872}
\pmtype{Topic}
\pmcomment{trigger rebuild}
\pmclassification{msc}{62-07}
\pmclassification{msc}{11-00}
\pmclassification{msc}{01A20}
\pmclassification{msc}{01A17}
\pmrelated{ArithmeticGeometricMeansInequality}
\pmrelated{GeneralMeansInequality}
\pmrelated{HeronianMeanIsBetweenGeometricAndArithmeticMean}
\pmrelated{IntegerContraharmonicMeans}
\pmrelated{OrderOfSixMeans}
\pmdefines{Pythagorean means}
\pmdefines{Babylonian inequality chain}

\endmetadata

% this is the default PlanetMath preamble.  as your knowledge
% of TeX increases, you will probably want to edit this, but
% it should be fine as is for beginners.

% almost certainly you want these
\usepackage{amssymb}
\usepackage{amsmath}
\usepackage{amsfonts}

% used for TeXing text within eps files
%\usepackage{psfrag}
% need this for including graphics (\includegraphics)
%\usepackage{graphicx}
% for neatly defining theorems and propositions
 \usepackage{amsthm}
% making logically defined graphics
%%%\usepackage{xypic}
\usepackage{pstricks}
\usepackage{pst-plot}

% there are many more packages, add them here as you need them

% define commands here

\theoremstyle{definition}
\newtheorem*{thmplain}{Theorem}

\begin{document}
If $u$ and $v$ are positive numbers and\, $u \le v$,\, then their {\em Pythagorean means}, viz. the harmonic mean $h(u,v)$,\, the geometric mean \,$g(u,v)$,\, the arithmetic mean \,$a(u,v)$\, and the contraharmonic mean \,$c(u,v)$, obey the \PMlinkname{order}{TotalOrder}
\begin{align}
u \;\le\; h(u,v) \;\le\; g(u,v) \;\le\; a(u,v) \;\le\; c(u,v) \;\le\; v.
\end{align}
The part
\begin{align}
u \;\le\; h(u,v) \;\le\; g(u,v) \;\le\; a(u,v) \;\le\; v
\end{align}
of (1) was known already by the ancient Babylonians.\, Therefore it may be called the\, {\em Babylonian inequality chain} (Horst Hischer). \\

The below diagram plots the means \,$h(x,1)$ in black,\, $g(x,1)$ in \PMlinkescapetext{blue},\, $a(x,1)$ in cyan and\, $c(x,1)$ in green for\, $0 \le x \le 1$.


\begin{center}
\begin{pspicture}(-1,-1)(5.9,5.9)
\psaxes[Dx=10,Dy=10]{->}(0,0)(-0.5,-0.5)(5.7,5.7)
\rput(-0.2,-0.2){$0$}
\rput(5,-0.3){$1$}
\rput(-0.2,5){$1$}
\rput(-0.2,2.5){$\frac{1}{2}$}
\rput(5.85,-0.1){$x$}
\rput(-0.1,5.85){$y$}
\psdot(5,5)
\psdot(5,0)
\psdot[linecolor=green](0,5)
\psdot[linecolor=cyan](0,2.5)
\psline[linestyle=dotted](5,0)(5,5)
\psline[linestyle=dotted](0,5)(5,5)
\psplot[linecolor=green]{0}{5}{x x mul 25 add x 5 add div}
\psplot[linecolor=red]{0}{5}{x x mul 25 add 2 div sqrt}
\psplot[linecolor=cyan]{0}{5}{x 5 add 2 div}
\psplot[linecolor=blue]{0}{5}{5 x mul sqrt}
\psplot[linecolor=black]{0}{5}{10 x mul x 5 add div}
\psplot[linecolor=yellow]{0}{5}{x 5 x mul sqrt add 5 add 3 div}
\psplot[linecolor=brown]{0}{5}{5 x sub 2 exp 100 add sqrt x add 5 sub 2 div}

\end{pspicture}
\end{center}




Note that the \PMlinkname{linear graph}{Slope} of the arithmetic mean is the common \PMlinkname{tangent}{TangentLine} all those curves in the point \,$(1,1)$, since here the derivatives of all functions have the value $\frac{1}{2}$.\, The same concerns the yellow graph of the Heronian mean of $x$ and $1$, similarly the red graph of the quadratic mean.

\begin{thebibliography}{8}
\bibitem{HH}{\sc Horst Hischer}: ``\PMlinkexternal{Viertausend Jahre Mittelwertbildung}{http://hischer.de/uds/forsch/preprints/hischer/Preprint98.pdf}''.\, --- {\em mathematica didactica} \textbf{25} (2002).\; See also \PMlinkexternal{this}{http://www.math.uni-sb.de/PREPRINTS/preprint126.pdf}.
\end{thebibliography}

%%%%%
%%%%%
\end{document}
