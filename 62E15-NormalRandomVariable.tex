\documentclass[12pt]{article}
\usepackage{pmmeta}
\pmcanonicalname{NormalRandomVariable}
\pmcreated{2013-03-22 11:54:20}
\pmmodified{2013-03-22 11:54:20}
\pmowner{Koro}{127}
\pmmodifier{Koro}{127}
\pmtitle{normal random variable}
\pmrecord{22}{30527}
\pmprivacy{1}
\pmauthor{Koro}{127}
\pmtype{Definition}
\pmcomment{trigger rebuild}
\pmclassification{msc}{62E15}
\pmclassification{msc}{60E05}
\pmclassification{msc}{05C50}
\pmclassification{msc}{34K05}
\pmsynonym{normal distribution}{NormalRandomVariable}
\pmsynonym{standard normal distribution}{NormalRandomVariable}
\pmsynonym{bell distribution}{NormalRandomVariable}
\pmsynonym{bell curve}{NormalRandomVariable}
\pmsynonym{Gaussian}{NormalRandomVariable}
\pmrelated{AreaUnderGaussianCurve}
\pmrelated{JointNormalDistribution}

\usepackage{amssymb}
\usepackage{amsmath}
\usepackage{amsfonts}
\usepackage{graphicx}
%%%%\usepackage{xypic}
\begin{document}
\PMlinkescapeword{normal}
\PMlinkescapeword{theory}
\PMlinkescapeword{terms}


For any real numbers $\mu$ and $\sigma > 0$, the
\emph{Gaussian probability distribution function} 
with mean $\mu$ and variance $\sigma^2$ is defined by
\[
f(x) = \frac{1}{\sqrt{2 \pi \sigma^2}} 
\exp \left( - \tfrac12 \left( \tfrac{x - \mu}{\sigma} \right) ^2 \right).
\]
When $\mu=0$ and $\sigma = 1$, 
it is usually called \emph{standard normal distribution}.

A random variable $X$ having distribution density $f$ is said to be a 
\emph{normally distributed random variable}, denoted by $X\sim N(\mu,\sigma^2)$. 
It has expected value $\mu$, 
and variance $\sigma^2$. 

\subsection*{Cumulative distribution function}

The cumulative distribution function of a standard normal variable,
often denoted by
\[
\Phi(z) = \frac{1}{\sqrt{2\pi}} \int_{-\infty}^z e^{-x^2/2} \, dx\,,
\]
cannot be calculated in closed form in terms of the elementary functions,
but its values are tabulated in most statistics books and \PMlinkname{here}{TableOfProbabilitiesOfStandardNormalDistribution},
and can be computed using most computer statistical packages and spreadsheets.

\subsection*{Uses of the Gaussian distribution}

The normal distribution is probably the most frequently used distribution. 
Its graph looks like a bell-shaped function, which is why it is often 
called \emph{bell distribution}. 

The normal distribution is important in probability theory and statistics.
Empircally, many observed distributions, 
such as of people's heights, test scores, 
experimental errors, are found to be more or less to be Gaussian.
And theoretically, the normal distribution arises as a limiting
distribution of averages of large numbers of samples, justified 
by the central limit theorem.

\begin{figure}
\includegraphics{bell-curve}
\caption{Graph of densities of the normal 
distribution for various values of the standard deviation $\sigma$}
\end{figure}

\subsection*{Properties}

\begin{center}
\begin{tabular}{c|c}
Mean & $\mu$ \\
Variance & $\sigma^2$ \\
Skewness & 0 \\
Kurtosis & 3 \\
\hline
Moment-generating function & $M_X(t) = \exp\bigl(\mu t + (\sigma t)^2/2 \bigr) $ \\
Characteristic function & $\phi_X(t) = \exp\bigl(\mu i t -(\sigma t)^2/2 \bigr) $
\end{tabular}
\end{center}

\begin{itemize}
\item
If $Z$ is a standard normal random variable,
then $X = \sigma Z + \mu$ is distributed as $N(\mu, \sigma^2)$,
and conversely.
\item
The sum of any finite number of independent normal variables is itself
a normal random variable.
\end{itemize}


\subsection*{Relations to other distributions}
\begin{enumerate}
\item
The standard normal distribution can be considered as a Student-t distribution
with infinite degrees of freedom.
\item
The square of the standard normal random variable is the chi-squared random variable of degree 1.  Therefore, the sum of squares of $n$ independent standard normal random variables is the chi-squared random variable of degree $n$.
\end{enumerate}
%%%%%
%%%%%
%%%%%
%%%%%
\end{document}
