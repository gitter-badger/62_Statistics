\documentclass[12pt]{article}
\usepackage{pmmeta}
\pmcanonicalname{Overdispersion}
\pmcreated{2013-03-22 14:30:34}
\pmmodified{2013-03-22 14:30:34}
\pmowner{CWoo}{3771}
\pmmodifier{CWoo}{3771}
\pmtitle{overdispersion}
\pmrecord{10}{36047}
\pmprivacy{1}
\pmauthor{CWoo}{3771}
\pmtype{Definition}
\pmcomment{trigger rebuild}
\pmclassification{msc}{62J12}
\pmdefines{dispersion parameter}

\endmetadata

% this is the default PlanetMath preamble.  as your knowledge
% of TeX increases, you will probably want to edit this, but
% it should be fine as is for beginners.

% almost certainly you want these
\usepackage{amssymb,amscd}
\usepackage{amsmath}
\usepackage{amsfonts}
\usepackage{tabls}
% used for TeXing text within eps files
%\usepackage{psfrag}
% need this for including graphics (\includegraphics)
%\usepackage{graphicx}
% for neatly defining theorems and propositions
%\usepackage{amsthm}
% making logically defined graphics
%%%\usepackage{xypic}

% there are many more packages, add them here as you need them

% define commands here
\begin{document}
\PMlinkescapeword{normal}

When applying the generalized linear model or GLM to the real world, a phenomenon called \emph{overdispersion} occurs when the observed variance of the data is larger than the predicted variance.  This is particularly apparent in the case of a Poisson regression model, where 
\begin{center}
predicted variance = predicted mean, 
\end{center}
or the binary logistic regression model, where 
\begin{center}
predicted variance = predicted mean(1- predicted mean).
\end{center}
A parameter, called the \emph{dispersion parameter}, $\phi$, is introducted to the model to lower this overdispersion effect.  

The GLM, with the inclusion of this dispersion parameter, has the following density function: 
$$f_{Y_i}(y_i\mid\theta_i)=\operatorname{exp}[\frac{y\theta_i-b(\theta_i)}{a(\phi)}+c(y,\phi)]$$

Dispersion parameters for some of the well known distributions from the exponential family are listed in the following table:

\begin{center}
\begin{tabular}{|c|c|c|}
\hline
distribution&notation&dispersion parameter $\phi$\\
\hline\hline
\PMlinkname{Normal}{NormalRandomVariable}&$N(\mu,\sigma^2)$&$\sigma^2$\\
\hline
\PMlinkname{Poisson}{PoissonRandomVariable}&$Poisson(\mu)$&1\\
\hline
\PMlinkname{Binomial}{BernoulliDistribution2}&$Bin(m,\pi)$&$\displaystyle{\frac{1}{m}}$\\
\hline
\PMlinkname{Gamma}{GammaRandomVariable}&$Gamma(\alpha,\lambda)$&$\displaystyle{\frac{1}{\alpha}}$\\
\hline
\end{tabular}
\end{center}

\par
\begin{thebibliography}{8}
\bibitem{hilbe} J. M. Hilbe, {\em Negative Binomial Regression}, Cambridge University Press, Cambridge (2007).
\bibitem{agresti} A. Agresti, {\em An Introduction to Categorical Data Analysis}, Wiley \& Sons, New York (1996).
\bibitem{mccullagh} P. McCullagh and J. A. Nelder, {\em Generalized Linear Models}, Chapman \& Hall/CRC, 2nd ed., London (1989).
\bibitem{dobson} A. J. Dobson, {\em An Introduction to Generalized Linear Models}, Chapman \& Hall, 2nd ed. (2001).
\end{thebibliography}
%%%%%
%%%%%
\end{document}
