\documentclass[12pt]{article}
\usepackage{pmmeta}
\pmcanonicalname{StratifiedSampling}
\pmcreated{2013-03-22 15:17:17}
\pmmodified{2013-03-22 15:17:17}
\pmowner{CWoo}{3771}
\pmmodifier{CWoo}{3771}
\pmtitle{stratified sampling}
\pmrecord{6}{37080}
\pmprivacy{1}
\pmauthor{CWoo}{3771}
\pmtype{Definition}
\pmcomment{trigger rebuild}
\pmclassification{msc}{62D05}
%\pmkeywords{strata}
\pmdefines{stratified random sampling}
\pmdefines{stratum}
\pmdefines{stratum weight}

\endmetadata

\usepackage{amssymb,amscd}
\usepackage{amsmath}
\usepackage{amsfonts}
\usepackage{tabls}
% used for TeXing text within eps files
%\usepackage{psfrag}
% need this for including graphics (\includegraphics)
%\usepackage{graphicx}
% for neatly defining theorems and propositions
%\usepackage{amsthm}
% making logically defined graphics
%%%\usepackage{xypic}

% define commands here
\begin{document}
\PMlinkescapeword{subdivisions}
\PMlinkescapeword{place}
\PMlinkescapeword{units}
\PMlinkescapeword{groups}
\PMlinkescapeword{order}
\PMlinkescapeword{decide}
\PMlinkescapeword{information}
\PMlinkescapeword{divide}
\PMlinkescapeword{contains}

In sampling surveys, it is sometimes a good idea to break up the
population into subdivisions before any sampling were to take place.
For example, from a population $U$ of automobile insurance policies,
the claim frequencies (loosely speaking, this is the ratio of the
number of claims to the number of policies in the $U$) are found in
the following table:
\begin{center}
\begin{tabular}{|c|c|c|}
\hline
male drivers & female drivers & all drivers \\
\hline $10\%$ & $7\%$ & $9\%$ \\
\hline
\end{tabular}
\end{center}
Suppose that we would like to take a sample from $U$ so that when
the sample units are divided up into male drivers and female
drivers, the respective sample claim frequencies are more or less
$10\%$ and $7\%$.  How would we do this?  If a simple random sample
is taken directly from the population, we may get the total claim
frequency (for all drivers) to be more or less $9\%$, but when the
sample is broken down into two groups by gender, we may no longer
guarantee that the claim frequencies by gender match (more or less)
those calculated from the population.  To insure that the sample
taken preserves claim frequencies by gender, we would take a
\emph{stratified sampling}.
\\\\
Formally, in \emph{stratified sampling}, the following steps are
taken, in order, from a population $U$ of $N$ units:
\begin{enumerate}
\item Decide what subdivisions are to be analyzed from
within $U$ and what information (or statistics) within the
subdivisions should be ``preserved''.  For example, if we want to
analyze our data by gender, then we would have two subdivisions to
study. If there is more than one categorical variable, then we would
look at all the \emph{possible} combinations of the these variables.
\item Make sure all the possible combinations are mutually exclusive
events;
\item Divide $U$ into $k$ subdivisions, or \emph{strata}, $U_i$,
where $k$ is the total number of possible combinations described
above.  From the first two steps, we have $$U=U_1\cup U_2\cup\ldots
U_k \mbox{ such that }U_i\cap U_j=\varnothing,$$ for all $i\neq j$
and $1\leq i,j\leq k$.  In addition, if we let $N_i=\lvert U_i
\rvert$, then
$$N=\sum_{i=1}^{k}N_i.$$
\item Draw a sample $S_i$ from each stratum $U_i$.
\end{enumerate}
\textbf{Remarks}.
\begin{itemize}
\item When each $S_i$ is a simple random sample within each $U_i$,
then we call this procedure a \emph{stratified random sampling}.
\item Each stratum corresponds to a number $$W_i:=\frac{N}{N_i},$$ called a
\emph{stratum weight}.
\item Suppose each sample $S_i$ contains $n_i$ units ($\lvert S_i
\rvert = n_i$) and that $n=\sum_{i=1}^{k}n_i$.  We call the
stratified sampling \emph{proportional} if, for each $i$,
$$\frac{n}{n_i}=W_i.$$
\end{itemize}
%%%%%
%%%%%
\end{document}
