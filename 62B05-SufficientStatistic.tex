\documentclass[12pt]{article}
\usepackage{pmmeta}
\pmcanonicalname{SufficientStatistic}
\pmcreated{2013-03-22 15:02:42}
\pmmodified{2013-03-22 15:02:42}
\pmowner{CWoo}{3771}
\pmmodifier{CWoo}{3771}
\pmtitle{sufficient statistic}
\pmrecord{11}{36759}
\pmprivacy{1}
\pmauthor{CWoo}{3771}
\pmtype{Definition}
\pmcomment{trigger rebuild}
\pmclassification{msc}{62B05}
\pmsynonym{sufficient estimator}{SufficientStatistic}
\pmsynonym{minimally sufficient statistic}{SufficientStatistic}
\pmsynonym{minimal sufficient}{SufficientStatistic}
\pmsynonym{minimally sufficient}{SufficientStatistic}
\pmdefines{minimal sufficient statistic}
\pmdefines{equivalent statistic}

\endmetadata

% this is the default PlanetMath preamble.  as your knowledge
% of TeX increases, you will probably want to edit this, but
% it should be fine as is for beginners.

% almost certainly you want these
\usepackage{amssymb,amscd}
\usepackage{amsmath}
\usepackage{amsfonts}

% used for TeXing text within eps files
%\usepackage{psfrag}
% need this for including graphics (\includegraphics)
%\usepackage{graphicx}
% for neatly defining theorems and propositions
%\usepackage{amsthm}
% making logically defined graphics
%%%\usepackage{xypic}

% there are many more packages, add them here as you need them

% define commands here
\begin{document}
Let $\lbrace f_\theta \rbrace$ be a statistical model with parameter
$\theta$.  Let $\boldsymbol{X}=(X_1,\ldots,X_n)$ be a random vector
of random variables representing $n$ observations.  A statistic $T=T(\boldsymbol{X})$ of $\boldsymbol{X}$ for the parameter $\theta$ is called a
\emph{sufficient statistic}, or a \emph{sufficient estimator}, if
the conditional probability distribution of $\boldsymbol{X}$ given
$T(\boldsymbol{X})=t$ is not a function of $\theta$ (equivalently,
does not depend on $\theta$).

In other words, all the information about the unknown parameter
$\theta$ is captured in the sufficient statistic $T$.  If, say, we
are interested in finding out the percentage of defective light
bulbs in a shipment of new ones, it is enough, or \emph{sufficient},
to count the number of defective ones (sum of the $X_i$'s), rather
than worrying about which individual light bulbs are the defective
ones (the vector $(X_1,\ldots,X_n)$).  By taking the sum, a certain
``reduction'' of data has been achieved.

\textbf{Examples}
\begin{enumerate}
\item Let $X_1,\ldots,X_n$ be $n$ independent observations from a
uniform distribution on integers $1,\ldots,\theta$.  Let
$T=\max\lbrace X_1,\ldots,X_n \rbrace$ be a statistic for $\theta$.
 Then the conditional probability distribution of
$\boldsymbol{X}=(X_1,\ldots,X_n)$ given $T=t$ is
$$P(\boldsymbol{X}\mid t)=\frac{P(X_1=x_1,\ldots,X_n=x_n,\max\lbrace X_n
\rbrace=t)}{P(\max\lbrace X_n \rbrace=t)}.$$ The numerator is $0$ if
$\max\lbrace x_n\rbrace\neq t$.  So in this case,
$P(\boldsymbol{X}\mid t)=0$ and is not a function of $\theta$.
Otherwise, the numerator is $\theta^{-n}$ and $P(\boldsymbol{X}\mid
t)$ becomes
$$\frac{\theta^{-n}}{P(\max\lbrace X_n \rbrace=t)}=
(\theta^nP(X_{(1)}\leq \cdots\leq X_{(n)}=t))^{-1},$$ where
$X_{(i)}$'s are the rearrangements of the $X_i$'s in a
non-decreasing order from $i=1$ to $n$.  For the denominator, we first note that
\begin{eqnarray*}
P(X_{(1)}\leq \cdots\leq X_{(n)}=t) &=& P(X_{(1)}\leq \cdots\leq X_{(n)}\leq t)-P(X_{(1)}\leq \cdots\leq X_{(n)}<t) \\ &=& P(X_{(1)}\leq \cdots\leq X_{(n)}\leq t)-P(X_{(1)}\leq \cdots\leq X_{(n)}\leq t-1).
\end{eqnarray*}
From the above equation, we find that there are $t^n-(t-1)^n$ ways to form non-decreasing finite sequences of $n$ positive integers such that the maximum of the sequence is
$t$.  So
$$(\theta^nP(X_{(1)}\leq \cdots\leq X_{(n)}=t))^{-1}=
(\theta^n(t^n-(t-1)^n)\theta^{-n})^{-1}=(t^n-(t-1)^n)^{-1}$$
 again is not a function of $\theta$.  Therefore, $T=\max\lbrace X_i\rbrace$ is a
 sufficient statistic for $\theta$.
 Here, we see that a reduction of data has been achieved by taking
 only the largest member of set of observations, not the entire set.
\item If we set $T(X_1,\ldots,X_n)=(X_1,\ldots,X_n)$, then we see
that $T$ is trivially a sufficient statistic for \emph{any}
parameter $\theta$.  The conditional probability distribution of
$(X_1,\ldots,X_n)$ given $T$ is 1.  Even though this is a sufficient
statistic by definition (of course, the individual observations
provide as much information there is to know about $\theta$ as
possible), and there is no loss of data in $T$ (which is simply a
list of all observations), there is really no reduction of data to
speak of here.
\item The sample mean
$$\overline{X}=\frac{X_1+\cdots+X_n}{n}$$
of $n$ independent observations from a normal distribution
$N(\mu,\sigma^2)$ (both $\mu$ and $\sigma^2$ unknown) is a
sufficient statistic for $\mu$.  This is the result of the
factorization criterion. Similarly, one sees that any partition of
the sum of $n$ observations $X_i$ into $m$ subtotals is a sufficient
statistic for $\mu$.  For instance,
$$T(X_1,\ldots,X_n)=(\sum_{i=1}^{j}X_i,\sum_{i=j+1}^{k}X_i,\sum_{i=k+1}^{n}X_i)$$
is a sufficient statistic for $\mu$.
\item Again, assume there are $n$ independent observations $X_i$ from
a normal distribution $N(\mu,\sigma^2)$ with unknown mean and
variance.  The sample variance
$$\frac{1}{n-1}\sum_{i=1}^{n}(X_i-\overline{X})^2$$ is \emph{not} a
sufficient statistic for $\sigma^2$.  However, if $\mu$ is a known
constant, then
$$\frac{1}{n-1}\sum_{i=1}^{n}(X_i-\mu)^2$$ is a sufficient statistic
for $\sigma^2$.
\end{enumerate}

A sufficient statistic for a parameter $\theta$ is called
a \emph{minimal sufficient statistic} if it can be expressed as a
function of any sufficient statistic for $\theta$.

\textbf{Example}. In example $3$ above, both the sample mean
$\overline{X}$ and the finite sum $S=X_1+\cdots+X_n$ are minimal
sufficient statistics for the mean $\mu$.  Since, by the
factorization criterion, any sufficient statistic $T$ for $\mu$ is a
vector whose coordinates form a partition of the finite sum, taking
the sum of these coordinates is just the finite sum $S$.  So, we
have just expressed $S$ as a function of $T$.  Therefore, $S$ is
minimal.  Similarly, $\overline{X}$ is minimal.

Two sufficient statistics $T_1,T_2$ for a parameter $\theta$ are
said to be equivalent provided that there is a bijection $g$ such
that $g\circ T_1=T_2$.  $\overline{X}$ and $S$ from the above
example are two equivalent sufficient statistics.  Two minimal sufficient statistics for the same parameter are equivalent.
%%%%%
%%%%%
\end{document}
