\documentclass[12pt]{article}
\usepackage{pmmeta}
\pmcanonicalname{HazardFunction}
\pmcreated{2013-03-22 14:27:45}
\pmmodified{2013-03-22 14:27:45}
\pmowner{CWoo}{3771}
\pmmodifier{CWoo}{3771}
\pmtitle{hazard function}
\pmrecord{6}{35982}
\pmprivacy{1}
\pmauthor{CWoo}{3771}
\pmtype{Definition}
\pmcomment{trigger rebuild}
\pmclassification{msc}{62N99}
\pmclassification{msc}{62P05}
\pmdefines{cumulative hazard function}

% this is the default PlanetMath preamble.  as your knowledge
% of TeX increases, you will probably want to edit this, but
% it should be fine as is for beginners.

% almost certainly you want these
\usepackage{amssymb,amscd}
\usepackage{amsmath}
\usepackage{amsfonts}

% used for TeXing text within eps files
%\usepackage{psfrag}
% need this for including graphics (\includegraphics)
%\usepackage{graphicx}
% for neatly defining theorems and propositions
%\usepackage{amsthm}
% making logically defined graphics
%%%\usepackage{xypic}

% there are many more packages, add them here as you need them

% define commands here
\begin{document}
Let $Y$ be a random variable with probability density function $f_Y(y)$.  Then the \emph{hazard function} $h(y)$ is defined to be:
$$h(y) = \frac{f_Y(y)}{1 - F_Y(y)} = \frac{f_Y(y)}{S(y)},$$
where $S(y)$ is the survivor function and $Y$ is the survival time.

The hazard function is the rate of probability of death (non survival) is changing at time $Y=y$, given survival up to time $y$:
$$h(y) = \lim_{\Delta y\rightarrow 0} \frac
{P(y\leq Y \leq y+\Delta y \mid Y > y)}{\Delta y}.$$

The \emph{cumulative hazard function}, $H(y)$ of $Y$ is defined as 
$$H(y) = \int_{-\infty}^{y} h(t) dt.$$

From this definition, we see that $H(y)=-\operatorname{ln}S(y)$.

\textbf{Examples}.
The hazard functions for the three most widely used probability density functions for survival time are:

\begin{itemize}
\item The exponential distribution, with $h(y)=\gamma$.
\item The Weibull distribution, with $h(y)=\gamma y^{\gamma-1}$ using the standard Weibull distribution.
\item The extreme-value distribution, with $h(y)=\frac{1}{\beta}\operatorname{exp}(\frac{y-\alpha}{\beta})$.
\end{itemize}
%%%%%
%%%%%
\end{document}
