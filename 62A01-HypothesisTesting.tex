\documentclass[12pt]{article}
\usepackage{pmmeta}
\pmcanonicalname{HypothesisTesting}
\pmcreated{2013-03-22 14:42:12}
\pmmodified{2013-03-22 14:42:12}
\pmowner{CWoo}{3771}
\pmmodifier{CWoo}{3771}
\pmtitle{hypothesis testing}
\pmrecord{5}{36316}
\pmprivacy{1}
\pmauthor{CWoo}{3771}
\pmtype{Definition}
\pmcomment{trigger rebuild}
\pmclassification{msc}{62A01}
\pmrelated{ChiSquaredStatistic}
\pmdefines{null hypothesis}
\pmdefines{alternative hypothesis}
\pmdefines{significance level}
\pmdefines{t test}
\pmdefines{f test}
\pmdefines{chi-squared test}
\pmdefines{chi-square test}
\pmdefines{$\chi^2$ test}
\pmdefines{p-value}

\endmetadata

% this is the default PlanetMath preamble.  as your knowledge
% of TeX increases, you will probably want to edit this, but
% it should be fine as is for beginners.

% almost certainly you want these
\usepackage{amssymb,amscd}
\usepackage{amsmath}
\usepackage{amsfonts}

% used for TeXing text within eps files
%\usepackage{psfrag}
% need this for including graphics (\includegraphics)
%\usepackage{graphicx}
% for neatly defining theorems and propositions
%\usepackage{amsthm}
% making logically defined graphics
%%%\usepackage{xypic}

% there are many more packages, add them here as you need them

% define commands here
\begin{document}
\PMlinkescapeword{mean}
\PMlinkescapeword{link}
\PMlinkescapeword{times}
\PMlinkescapeword{types}
\PMlinkescapeword{normal}

Hypothesis testing is a statistical inferencial procedure in which a statement based on some experimental or observational study is formulated, tested, then put through a decision process.  The decision process either accepts or rejects the statement.
\par
More precisely, the hypothesis testing procedure can be broken down into three steps:
\begin{enumerate}
\item Formulation of a (hypothetical) statement. \par
The hypothetical statement formed is called the null hypothesis, or $H_0$.  For example, in testing whether a coin is ``fair'', it is tossed 100 times and the number of heads are counted.  $H_0$ could be $\lbrace\mbox{ The number of heads = }50 \rbrace$.  Accompanying the null hypothesis $H_0$ is the alternative hypothesis $H_a$.  The statement in $H_a$ is the compliment of the statement in $H_0$ (the universe is the sample space).  For example, $H_a$ would be $\lbrace\mbox{ The number of heads }\neq 50 \rbrace$.
\item Testing of the statement. \par
This is usually the most mathematical part of the procedure.  To test $H_0$, first assume $H_0$ is true.  Then apply an appropriate test statistic using values obtained from the study.  There are many test statistics, depending on $H_0$, $H_a$, and the nature of the study.  \par Based on the distributional forms of these test statistics, four major types of tests are of interest.  A \emph{t test} is based on a test statistic that has a t-distribution.  An \PMlinkname{\emph{f test}}{fdistribution} and a \PMlinkname{$\chi^2$ \emph{test}}{ChiSquaredRandomVariable} are so named for the same reason.  A \emph{z-test} is one that is based on a test statistic having a normal or Gaussian distribution. \par Before calculating the test statistic, a value of the \emph{significance level} of the test needs to be specified.  The significance level, known as $\alpha$, is the probability of rejecting $H_0$ (or accepting $H_a$) when in fact, $H_0$ is true: $\alpha=P(H_a\mid H_0)$.
\item Deciding whether to accept or reject the statement. \par
Once the value of the test statistic is obtained, it is used to find a corresponding probability of obtaining such a statistic given that $H_0$ is true.  This probability is called the \emph{p-value}.  This $p$-value is then compared $\alpha$, the significance level of the test.  If $p$-value $<\alpha$, then the usual next step is to reject the null hypothesis $H_0$ (and $H_a$ accepted).  Otherwise, $H_0$ will be accepted.
\end{enumerate}
When a statement (whether it is null hypothesis or the alternative hypothesis) is accepted, it merely says that, statistically, there is not enough evidence to reject the statement.  Acceptance of a hypothetical statement does not prove that the underlying statement is true.
\par
The concept of statistical hypothesis testing can be found in any standard introductory statistics textbooks, as well as numerous internet websites (for example, \PMlinkexternal{click to find the result of a Google search}{http://www.google.com/search?hl=en&lr=&q=hypothesis+testing}).  The purpose of this entry is to give a very brief description of hypothesis testing and to serve as a link reference for other entries.
%%%%%
%%%%%
\end{document}
