\documentclass[12pt]{article}
\usepackage{pmmeta}
\pmcanonicalname{LehmerMean}
\pmcreated{2013-03-22 19:02:06}
\pmmodified{2013-03-22 19:02:06}
\pmowner{pahio}{2872}
\pmmodifier{pahio}{2872}
\pmtitle{Lehmer mean}
\pmrecord{11}{41908}
\pmprivacy{1}
\pmauthor{pahio}{2872}
\pmtype{Definition}
\pmcomment{trigger rebuild}
\pmclassification{msc}{62-07}
\pmclassification{msc}{11-00}
\pmrelated{OrderOfSixMeans}
\pmrelated{LeastAndGreatestNumber}
\pmrelated{MinimalAndMaximalNumber}

\endmetadata

% this is the default PlanetMath preamble.  as your knowledge
% of TeX increases, you will probably want to edit this, but
% it should be fine as is for beginners.

% almost certainly you want these
\usepackage{amssymb}
\usepackage{amsmath}
\usepackage{amsfonts}

% used for TeXing text within eps files
%\usepackage{psfrag}
% need this for including graphics (\includegraphics)
%\usepackage{graphicx}
% for neatly defining theorems and propositions
 \usepackage{amsthm}
% making logically defined graphics
%%%\usepackage{xypic}

% there are many more packages, add them here as you need them

% define commands here

\theoremstyle{definition}
\newtheorem*{thmplain}{Theorem}

\begin{document}
Let $p$ be a real number.\, \emph{Lehmer mean} of the positive numbers $a_1,\,\ldots,\,a_n$ is defined as
\begin{align}
L_p(a_1,\,\ldots,\,a_n) \;:=\; \frac{a_1^p+\ldots+a_n^p}{a_1^{p-1}+\ldots+a_n^{p-1}}.
\end{align}
This definition fulfils both requirements set for \PMlinkname{means}{Mean3}.\, In the case of Lehmer mean of two positive numbers $a$ and $b$ we see for\, $a \leqq b$\, that
$$a \;=\; \frac{a^p\!+\!ab^{p-1}}{a^{p-1}\!+\!b^{p-1}} \;\leqq\; \frac{a^p\!+\!b^p}{a^{p-1}\!+\!b^{p-1}} 
\;\leqq\; \frac{a^{p-1}b\!+\!b^p}{a^{p-1}\!+\!b^{p-1}} \;=\; b.$$


The Lehmer mean of certain numbers is the greater the greater is the parametre $p$, i.e. 
$$L_p(a_1,\,\ldots,\,a_n) \;\geqq\; L_q(a_1,\,\ldots,\,a_n) \quad \forall\; p \;>\; q.$$
This turns out from the nonnegativeness of the partial derivative of $L_p$ with respect to $p$; in the case \,$n =2$\, it writes
$$\frac{\partial L_p}{\partial p} \;=\; \frac{a^{p-1}b^{p-1}(a\!-\!b)(\ln{a}-\ln{b})}{(a^{p-1}\!+\!b^{p-1})^2} 
\;\geqq\; 0.$$
Thus in the below list containing special cases of Lehmer mean, the \PMlinkescapetext{harmonic mean} is the least and the contraharmonic the greatest (cf. the comparison of Pythagorean means).

E.g. for two arguments $a$ and $b$, one has
\begin{itemize}
\item $\displaystyle L_0(a,\,b) \,=\, \frac{2ab}{a\!+\!b}$, \quad harmonic mean,
\item $\displaystyle L_{1/2}(a,\,b) \,=\, \sqrt{ab}$, \quad geometric mean,
\item $\displaystyle L_1(a,\,b) \,=\, \frac{a\!+\!b}{2}$, \quad arithmetic mean,
\item $\displaystyle L_2(a,\,b) \,=\, \frac{a^2\!+\!b^2}{a\!+\!b}$, \quad contraharmonic mean.
\end{itemize}



\textbf{Note.}\, The \PMlinkname{least}{LeastNumber} and the \PMlinkname{greatest of the numbers}{GreatestNumber} 
$a_1,\,\ldots,\,a_n$ may be regarded as borderline cases of the Lehmer mean, since
$$\lim_{p\to-\infty}L_p(a_1,\,\ldots,\,a_n) \;=\; \min\{a_1,\,\ldots,\,a_n\}, \quad 
\lim_{p\to+\infty}L_p(a_1,\,\ldots,\,a_n) \;=\; \max\{a_1,\,\ldots,\,a_n\}.$$
For proving these equations, suppose that there are exactly $k$ greatest (resp. least) ones among the numbers and that those are \,$a_1 = \ldots = a_k$.\, Then we can write
$$L_p(a_1,\,\ldots,\,a_n) \;=\; 
\frac{a_1^p\left[k+\!\left(\frac{a_{k+1}}{a_1}\right)^p\!+\ldots+\!\left(\frac{a_{n}}{a_1}\right)^p\right]}
 {a_1^{p-1}\left[k+\!\left(\frac{a_{k+1}}{a_1}\right)^{p-1}\!+\ldots+\!\left(\frac{a_{n}}{a_1}\right)^{p-1}\right]}.$$
Letting\, $p \to +\infty$\, (resp. $p \to -\infty$),\, this equation yields
$$L_p(a_1,\,\ldots,\,a_n) \;\longrightarrow\; a_1.$$
%%%%%
%%%%%
\end{document}
